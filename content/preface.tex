\begin{flushright}
	\zihao{1} 前言
\end{flushright}

软件世界每天都在发展,CMake也不例外。经过20多年的不断发展,现在是构建C++应用程序的行业标准。虽然CMake具有非常丰富的功能,但其文档中的例子和指南实在是太少了。而这,恰恰就是CMake实践的最佳切入点。

本书不会去解释CMake的细节和特性,而会通过一些例子来说明在构建软件时,如何使用CMake完成各种任务(而不是覆盖每种边缘情况)。本书的目的是让事情尽可能简单,同时涵盖完成事情的推荐的最佳实践,从而读者只需要知道自己需要的CMake功能即可。

我们将尝试先解释概念,然后用例子来说明。通过这种方式,通过实践进行学习,并且可将这些知识应用到日常工作中。估计本书的读者大多是工程师,所以我们对书的内容进行了相应的调整。写这本书的时候,我们首先扮演的角色是软件工程师,然后才是作者。因此,本书的内容会更实用,而不仅停留在理论层面,从而证明了相应的技术可以在日常工作中使用。

从工程师中来,再回到工程师中去。我们希望你享受这本书。

\hspace*{\fill} \\ %插入空行
\textbf{适宜读者}

这本书针对软件工程师和构建系统维护人员,他们经常使用C或C++,从而可以使用CMake来减轻日常任务的重担。基本的C++和编程知识有助于读者更好地理解书中的例子。

\hspace*{\fill} \\ %插入空行
\textbf{本书内容}

Chapter 1, Kickstarting CMake, explains what CMake is in a nutshell and then jumps right into installing CMake and building something with CMake. You will learn how to install the latest stable version manually, even if it's not provided by your package manager. You'll also learn about the basic concepts behind CMake and why it is a build system generator and not a build system itself. You will learn how it fits into modern software development with C++ (and C).

Chapter 2, Accessing CMake in the Best Ways, shows how to best use CMake from the command line with a GUI and how CMake integrates with some common IDEs and editors.

Chapter 3, Creating a CMake Project, takes you through setting up a project to build an executable and a library and linking the two together.

Chapter 4, Packaging, Deploying, and Installing a CMake Project, shows you how to create a distributable version of your software project. You will learn how to add installation instructions and package the project using CMake and CPack (CMake's packaging program).

Chapter 5, Integrating Third-Party Libraries and Dependency Management, explains how to integrate existing third-party libraries into your project. It also shows you how to add libraries already installed on your system, external CMake projects, and non-CMake projects.

Chapter 6, Automatically Generating Documentation, explores how to generate documentation from your code with CMake as part of the build process with doxygen, dot (graphviz), and plantuml.

Chapter 7, Seamlessly Integrating Code-Quality Tools with CMake, shows you how to integrate unit testing, code sanitizers, static code analysis, and code coverage tools into your project. It will show you how CMake can help to discover and execute tests.

Chapter 8, Executing Custom Tasks with CMake, explains how you can integrate almost any tool into your build process. You will learn how to wrap external programs into custom targets or hook them into the build process to execute them. We will cover how custom tasks can be used to generate files and how they can consume files produced by other targets. You will also learn how to execute system commands during the configuration of the CMake build and how to create platform-agnostic commands using the CMake script mode.

Chapter 9, Creating Reproducible Build Environments, shows how you can build an environment portable between various machines including CI/CD pipelines, and how to work with Docker, sysroots, and CMake presets to make your build work "out of the box" everywhere.

Chapter 10, Handling Big Projects and Distributed Repositories in a Superbuild, simplifies managing projects that are distributed across multiple git repositories with CMake. You will learn how to create a super-build that allows you to build specific versions as well as the latest nightly builds. You will explore what prerequisites it needs and how to combine them.

Chapter 11, Automated Fuzzing with CMake, shows how you can integrate and use fuzzing tools with CMake.

Chapter 12, Cross-Platform Compiling and Custom Toolchains, demonstrates how you can use cross-platform toolchains. You will also learn how to write your own toolchain definitions and conveniently use different toolchains with CMake.

Chapter 13, Reusing CMake Code, explains CMake modules and how you can generalize your CMake files. You will learn how to write broadly used modules, which you can ship individually from your project.

Chapter 14, Optimizing and Maintaining CMake Projects, suggests how to get faster build times and provides tips and tricks for keeping a CMake project neat and tidy over a long period of time. 

Chapter 15, Migrating to CMake, explains a high-level strategy on how to migrate a large existing codebase to CMake without the need to completely stop your development.

Chapter 16, Contributing to CMake and Further Reading Material, suggests where to go if you want to contribute, what is looked for, and basic contributing guidelines. It will also guide you on where to find additional in-depth information or more specific literature.

\hspace*{\fill} \\ %插入空行
\textbf{编译环境}

将CMake 3.21或更新版本,以及安装的至少支持C++14的现代C++编译器。一些例子可能需要额外的软件来运行,在相关章节中会提到。示例中使用的所有软件都是开源的,可以免费获得。

\begin{table}[H]
	\centering
	\begin{tabular}{|l|l|}
		\hline
		书中涉及的软件/硬件                                                                                                                  & 操作系统                                                             \\ \hline
		CMake 3.13.4或更高版本                                                                                                                                  &                                                                           Linux, Windows或macOS       \\  \hline
		GCC、Clang或MSVC              &                                                                                  Linux, Windows或macOS\\  \hline
		Git  &  Linux, Windows或macOS                                                                                \\ \hline
	\end{tabular}
\end{table}

\textbf{如果您正在使用这本书的数字版本,建议您自己输入代码或从本书的GitHub存储库访问代码(下一节有链接)。这样做将帮助您避免与复制和粘贴代码相关的错误。}

\hspace*{\fill} \\ %插入空行
\textbf{下载示例}

可以从GitHub上下载这本书的示例代码文件\url{https://github.com/PacktPublishing/CMake-Best-Practices}。若代码有更新,会在GitHub存储库中更新。

还可以从丰富的书籍和视频目录中获得其他代码包\url{https://github.com/PacktPublishing/}。

\hspace*{\fill} \\ %插入空行
\textbf{下载彩图}

我们还提供了一个PDF文件,其中有本书中使用的屏幕截图和图表的彩图。可在此下载: \url{https://static.packt-cdn.com/downloads/9781803239729_ColorImages.pdf}

\hspace*{\fill} \\ %插入空行
\textbf{联系方式}

我们欢迎读者的反馈。

反馈:如果你对这本书的任何方面有疑问,需要在你的信息的主题中提到书名,并给我们发邮件到\url{customercare@packtpub.com}。

勘误:尽管我们谨慎地确保内容的准确性,但错误还是会发生。如果您在本书中发现了错误,请向我们报告,我们将不胜感激。请访问\url{www.packtpub.com/support/errata},选择相应书籍,点击勘误表提交表单链接,并输入详细信息。

盗版:如果您在互联网上发现任何形式的非法拷贝,非常感谢您提供地址或网站名称。请通过\url{copyright@packt.com}与我们联系,并提供材料链接。

如果对成为书籍作者感兴趣:如果你对某主题有专长,又想写一本书或为之撰稿,请访问\url{authors.packtpub.com}。

\hspace*{\fill} \\ %插入空行
\textbf{分享你的想法}

阅读完本书后,我们很想听听你的想法!请点击这里直接进入这本书的亚马逊评论页面并分享你的反馈。

您的评论对我们和科技界都很重要,将帮助我们确保提供的内容是高质量的。

\newpage






















