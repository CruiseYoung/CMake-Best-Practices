In this chapter, you were provided with a brief overview of CMake. First, you learned how to install and run a simple build. Then, you learned about the two-stage build process of CMake before touching on the most important language features for writing CMake files.

By now, you should be able to build the examples provided in this book's GitHub repository: \url{https://github.com/PacktPublishing/CMake-BestPractices}|. You learned about the core features of the CMake language such as variables, targets, and policies. We briefly covered functions and macros, as well as conditional statements and loops for flow control. As you continue reading this book, you will use what you have learned so far to discover further good practices and techniques to move from simple one-target projects to complex software projects that keep being maintainable through a good CMake setup.

In the next chapter, we will learn how some of the most common tasks in CMake can be performed and how CMake works together with various IDEs.