CMake is freely available to download from \url{https://cmake.org/download/}. It is available as either a precompiled binary or as source code. For most use cases, the precompiled binary is fully sufficient, but since CMake itself has very few dependencies, building a version is also possible.

Any major Linux distribution offers CMake over its package repositories. Although the pre-packaged versions of CMake are not usually the latest releases, these installations are often sufficient to use if the system is regularly updated.

\begin{tcolorbox}[colback=webgreen!5!white,colframe=webgreen!75!black,title=Note]
The minimum version of CMake to use with the examples in this book is 3.21. We recommend that you download the appropriate version of CMake manually to ensure that you get the correct version.
\end{tcolorbox}

\subsubsubsection{1.3.1\hspace{0.2cm}Building CMake from source}

CMake is written in C++ and uses Make to build itself. Building CMake from scratch is possible, but for most use cases, using the binary downloads will do just fine.

After downloading the source package from \url{https://cmake.org/download/}, extract it to a folder and run the following command:

\begin{tcblisting}{commandshell={}}
./configure make
\end{tcblisting}

If you want to build cmake-gui as well, configure it with the -{}-qt-gui option. This requires Qt to be installed. Configuring will take a while, but once it's succeeded, CMake can be installed using the following command:

\begin{tcblisting}{commandshell={}}
make install
\end{tcblisting}

To test whether the installation was successful, you can execute the following command:

\begin{tcblisting}{commandshell={}}
cmake --version
\end{tcblisting}

This will print out the version of CMake, like this:

\begin{tcblisting}{commandshell={}}
cmake version 3.21.2
CMake suite maintained and supported by Kitware (kitware.com/cmake).
\end{tcblisting}








































































