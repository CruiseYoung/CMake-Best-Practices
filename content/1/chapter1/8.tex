A common problem when building software with CMake is how to share good or working configurations to build a project. Often, people and teams have a preferred way of where the build artifacts should go, which generator to use on which platform, or just the desire that the CI environment should use the same settings to build as it does locally. Since CMake 3.19 came out in December 2020, this information can be stored in CMakePresets.json files, which are placed in the root directory of a project. Additionally, each user can superimpose their configuration with a CMakeUserPresets.json file. The basic presets are usually placed under version control, but the user presets are not checked into the version system. Both files follow the same JSON format, with the top-level outline being as follows:

\begin{lstlisting}[style=styleCMake]
{
"version": 3,
"cmakeMinimumRequired": {
	"major": 3,
	"minor": 21,
	"patch": 0
},
"configurePresets": [...],
"buildPresets": [...],
"testPresets": [...]
}
\end{lstlisting}

\begin{enumerate}
\item 
The first line, "version": 3, denotes the schema version of the JSON file. CMake 3.21 supports up to version 3, but it is expected that new releases will bring new versions of the schema.

\item 
Next, cmakeMinimumRequired{...} specifies which version of CMake to use. Although this is optional, it is good practice to put this in here and match the version with the one specified in the CMakeLists.txt file.

\item 
After that, the various presets for the different build stages can be added with configurePresets, buildPresets, and testPresets. As the name suggests, configurePresets applies to the configure stage of CMake's build process, while the other two are used for the build and test stages. The build and test presets may inherit one or more configure presets. If no inheritance is specified, they apply to all the previous steps.
\end{enumerate}

To see what presets have been configured in a project, run cmake -{}-list-presets to see a list of available presets. To build using a preset, run cmake -{}-build -{}-preset name.

To see the full specification of the JSON schema, go to \url{https://cmake.org/cmake/help/v3.21/manual/cmake-presets.7.html}.

Presets are a good way to share knowledge about how to build a project in a very explicit way. At the time of writing, more and more IDEs and editors are adding support for CMake presets natively, especially for handling cross-compilation with toolchains. Here, we're only giving you the briefest overview of CMake presets; they will be covered in more depth in Chapter 12, Cross-Platform Compiling and Custom Toolchains.









































