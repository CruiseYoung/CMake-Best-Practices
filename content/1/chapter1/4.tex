现在,试试安装的CMake是否有效。我们提供了一个\texttt{hello world}示例,可以立即下载,并进行构建。打开一个控制台,输入以下内容:

\begin{tcblisting}{commandshell={}}
git clone https://github.com/PacktPublishing/CMake-BestPractices.git
cd CMake-Best-Practices/chapter_1
mkdir build
cd build
cmake ..
cmake -–build .
\end{tcblisting}

这将产生名为\texttt{Chapter\_1}的可执行文件,在控制台上输出\texttt{Welcome to CMake Best Practices}。

详细看看这里发生了什么:

\begin{enumerate}
\item 
首先,使用Git签出示例存储库,然后创建构建文件夹。示例CMake项目的文件结构在构建前是这样的:

\begin{tcblisting}{commandshell={}}
.
├── CMakeLists.txt
└── build
└── src
      └── main.cpp
\end{tcblisting}

除了源代码的文件夹外,还有一个名为CMakeLists.txt的文件。这个文件包含CMake关于如何创建项目的构建说明,以及如何构建的说明。每个CMake项目在项目的根目录下都有一个CMakeLists.txt文件,但是在不同的子文件夹中可能有许多同名的文件。

\item 
下载存储库之后,使用\texttt{cmake}启动构建过程。CMake的构建过程分为两个阶段。第一步称为配置,读取CMakeLists.txt文件,并为系统的本地构建工具链生成指令。第二步,执行这些构建指令,并构建可执行文件或库。

配置步骤中,将检查构建需求,解析依赖关系,并生成构建指令。

\item 
项目还会创建一个名为CMakeCache.txt的文件,其中包含创建构建指令所需的信息。\texttt{cmake -{}-build}的下一个阶段,通过内部调用CMake来执行构建;若在Windows上,则通过调用Visual Studio编译器来实现,这是编译二进制文件的实际方式。若一切顺利,在构建文件夹中会有一个名为\texttt{Chapter1}的可执行文件。
\end{enumerate}

简单起见,前面的示例中进入构建目录,并使用相对路径来查找源文件夹。这通常很方便,但想从其他地方使用CMake,可以使用\texttt{-{}-S}选项来选择源文件,以及使用\texttt{-{}-B}选项来指定构建文件夹:

\begin{tcblisting}{commandshell={}}
cmake -S /path/to/source -B /path/to/build
cmake -build /path/to/build
\end{tcblisting}

在持续集成环境中使用CMake时,显式传递构建目录和源目录通常很方便,也助于提高可维护性。若想为不同的配置创建不同的构建目录,例如在构建跨平台软件时,这也会很有帮助。

\subsubsubsection{1.4.1\hspace{0.2cm}简单的CMakeLists.txt示例}

对于非常简单的\texttt{hello world}例子,CMakeLists.txt文件只包含几行指令:

\begin{lstlisting}[style=styleCMake]
cmake_minimum_required(VERSION 3.21)
project(
  "chapter1"
  VERSION 1.0
  DESCRIPTION "A simple project to demonstrate basic CMake
    usage" LANGUAGES CXX)

add_executable(Chapter1)
target_sources(Chapter1 PRIVATE src/main.cpp)
\end{lstlisting}

更详细地说明一下:

\begin{itemize}
\item 
第一行定义了构建这个项目所需的CMake的最低版本。每个CMakeLists.txt文件都以这个指令开始。这是用来提示用户,若项目使用了CMake的特性,可能只有在某个版本以上才可用。通常,建议将版本设置为支持项目中使用特性的最低版本。

\item 
下一行是要构建的项目的名称、版本和描述,后面是项目中使用的编程语言。这里,使用\texttt{CXX}将其标记为一个C++项目。

\item 
\texttt{add\_executable}告诉CMake要构建一个可执行文件(与库或自定义目标不同)。

\item 
\texttt{target\_sources}告诉CMake在哪里查找名为\texttt{Chapter1}的可执行文件的源文件,并且源文件的可见性仅限于可执行文件。
\end{itemize}

祝贺你——现在可以用CMake创建程序了。为了理解这些指令背后发生了什么,需要继续了解一下CMake的构建过程。































