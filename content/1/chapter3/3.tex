首先,我们将从一个简单的hello world C++程序创建一个简单的可执行文件。下面的C++程序将打印出\texttt{Welcome to CMake Best Practices}:

\begin{lstlisting}[style=styleCXX]
#include <iostream>
int main(int, char **) {
	std::cout << "Welcome to CMake Best Practices\n";
	return 0;
}
\end{lstlisting}

要构建此文件,需要编译它并为可执行文件指定一个名称。来看看CMakeLists.txt文件是怎样构建这个可执行文件的:

\begin{lstlisting}[style=styleCMake]
cmake_minimum_required(VERSION 3.21)

project(
	hello_world_standalone
	VERSION 1.0
	DESCRIPTION"A simple C++ project"
	HOMEPAGE_URL https://github.com/PacktPublishing/CMake-BestPractices
	LANGUAGES CXX
)

add_executable(hello_world)
target_sources(hello_world PRIVATE src/main.cpp)
\end{lstlisting}

第一行,\texttt{cmake\_minimum\_required(VERSION 3.21)},期望看到的CMake的版本,以及CMake将启用哪些特性。对于本书,所有的例子中都使用CMake 3.21,但是出于兼容性的原因,读者们可以选择一个较低的版本。

对于本例,3.1版本将是绝对的最小值,因为在此之前,\texttt{target\_sources}不可用。将\texttt{cmake\_minimum\_required}指令放在每个CMakeLists.txt文件的顶部是一个很好的做法。

接下来,使用\texttt{project()}指令设置项目。第一个参数是项目的名称——我们的例子中为“hello\_world\_standalone”。

接下来,项目的版本设置为1.0版本。下面是一个简短的描述和主页的URL。最后,\texttt{LANGUAGES CXX}属性指定正在构建一个C++项目。除了项目名称之外,所有参数都可选。

调用\texttt{add\_executable(hello\_world)}指令,会创建一个名为\texttt{hello\_world}的目标。这也将是这个目标创建的可执行文件名。

现在已经创建了目标,使用\texttt{target\_sources}完成了向目标添加C++源文件。\texttt{Chapter3}是目标名,在\texttt{add\_executable}中指定。\texttt{PRIVATE}定义源仅用于构建此目标,而不用于依赖的目标。在范围说明符之后,有一个相对于当前CMakeLists.txt文件路径的源文件列表。如果需要,当前处理的CMakeLists txt文件的位置可以通过\texttt{CMAKE\_CURRENT\_SOURCE\_DIR}得到。

源码可以直接添加到\texttt{add\_executable},也可以单独使用\texttt{target\_sources},将它们与\texttt{target\_sources}一起添加。通过使用\texttt{PRIVATE}、\texttt{PUBLIC}或\texttt{INTERFACE},可以显式地定义在何处使用源码。但是,指定\texttt{PRIVATE}以外的内容只对库目标有意义。

经常看到的一种常见模式是用项目名称来命名项目的主可执行文件:

\begin{lstlisting}[style=styleCMake]
project(hello_world
...
)
add_executable(${PROJECT_NAME})
\end{lstlisting}

乍一看这似乎很方便,但不推荐这样做。项目名称和目标具有不同的语义含义,因此应该将其独立对待,因此应该避免使用\texttt{PROJECT\_NAME}作为目标的名称。

可执行文件非常重要,而且非常容易创建,除非构建的是一个庞然大物,否则应该使用库来模块化和分发代码。下一节中,我们将学习如何构建库,以及如何处理不同的链接方法。





























