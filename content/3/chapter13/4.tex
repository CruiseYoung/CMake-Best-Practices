
上一节中,我们学习了如何使用函数和宏在CMake项目中提供有用的功能。现在,来学习如何将这些函数和宏移动到单独的CMake模块中。

创建和使用简单的CMake模块文件非常简单:

\begin{enumerate}
\item 
创建一个<module\_name>.cmake文件。

\item 
<module\_name>.cmake文件中定义宏/函数。

\item 
将<module\_name>.cmake文件包含在必要的文件当中
\end{enumerate}

按照这些步骤一起创建一个模块,作为上一个git分支名称示例的后续,我们扩展范围编写一个CMake模块,该模块通过使用git命令提供检索分支名称、HEAD提交哈希、当前作者名称和当前作者电子邮件信息的能力。这一部分中,将使用chapter13/ex01\_git\_utility的示例。示例文件夹包含一个CMakeLists.txt文件和一个git.cmake文件,文件存放在cmake文件夹下。先来看看cmake/git.cmake:

\begin{lstlisting}[style=styleCMake]
# …
include_guard(DIRECTORY)
macro(git_get_branch_name result_var_name)
	execute_process(
		COMMAND git symbolic-ref -q --short HEAD
		WORKING_DIRECTORY "${CMAKE_CURRENT_SOURCE_DIR}"
		OUTPUT_VARIABLE ${result_var_name}
		OUTPUT_STRIP_TRAILING_WHITESPACE
		ERROR_QUIET
	)
endmacro()
# … git_get_head_commit_hash(), git_get_config_value()
\end{lstlisting}

git.cmake文件是cmake实用工具模块文件,包含三个宏,分别名为git\_get\_branch\_name、git\_get\_head\_commit\_hash和git\_get\_config\_value。此外,文件的顶部有include\_guard(DIRECTORY)。这类似于C/C++中的\#pragma once预处理器指令,防止文件多次包含。DIRECTORY参数表示include\_guard在目录范围内定义,该文件最多只能在当前目录内或以下包含一次。另外,可以指定GLOBAL参数(而不是DIRECTORY),以限制只包含一次文件,而不考虑范围。

看看如何使用git.cmake模块文件,研究下chapter13/ex01\_git\_utility中的CMakeLists.txt文件:

\begin{lstlisting}[style=styleCMake]
cmake_minimum_required(VERSION 3.21)
project(
	ch13_ex01_git_module
	VERSION 1.0
	DESCRIPTION "Chapter 13 Example 01, git utility module example"
	LANGUAGES CXX)
# Include the git.cmake module.
# Full relative path is given, since cmake/ is not in the 
# CMAKE_MODULE_PATH
include(cmake/git.cmake)
git_get_branch_name(current_branch_name)
git_get_head_commit_hash(current_head)
git_get_config_value("user.name" current_user_name)
git_get_config_value("user.email" current_user_email)

message(STATUS "-----------------------------------------")
message(STATUS "VCS (git) info:")
message(STATUS "\tBranch: ${current_branch_name}")
message(STATUS "\tCommit hash: ${current_head}")
message(STATUS "\tAuthor name: ${current_user_name}")
message(STATUS "\tAuthor e-mail: ${current_user_email}")
message(STATUS "-----------------------------------------")
\end{lstlisting}

通过指定模块文件的完整相对路径,CMakeLists.txt文件包含git.cmake文件。模块提供的git\_get\_branch\_name、git\_get\_head\_commit\_hash和git\_get\_config\_value宏分别用于检索分支名称、提交哈希、作者名和电子邮件到current\_branch\_name、current\_head、current\_user\_name和current\_user\_email。最后,通过\texttt{message(…)}指令将这些变量打印在屏幕上。配置示例项目,看看刚编写的git模块是否能如预期的那样工作:

\begin{tcblisting}{commandshell={}}
cd chapter13/ex01_git_utility/
cmake -S ./ -B ./build
\end{tcblisting}

命令的输出应该如下所示:

\begin{tcblisting}{commandshell={}}
-- The CXX compiler identification is GNU 9.4.0
-- Detecting CXX compiler ABI info
-- Detecting CXX compiler ABI info - done
-- Check for working CXX compiler: /usr/bin/c++ - skipped
-- Detecting CXX compile features
-- Detecting CXX compile features - done
-- -------------------------------------------
-- VCS (git) info:
-- Branch: chapter-development/chapter13
-- Commit hash: 1d5a32649e74e4132e7b66292ab23aaeed327fdc
-- Author name: Mustafa Kemal GILOR
-- Author e-mail: mustafagilor@gmail.com
-- -------------------------------------------
-- Configuring done
-- Generating done
-- Build files have been written to:
/home/toor/workspace/CMake-Best-Practices/chapter13
/ex01_git_utility/build
\end{tcblisting}

正如我们所看到的,已经成功地从git命令中检索了信息。我们的第一个CMake模块工作正常。

\subsubsubsection{13.4.1\hspace{0.2cm}案例研究——处理项目元数据文件}

假设有一个每行包含键值对的环境文件,项目中包含一些关于项目的元数据(例如,项目版本和依赖项)的外部文件并不罕见。该文件可以是不同的格式,例如:JSON或换行分隔的“键-值”对。当前的任务是创建一个实用工具模块,该模块读取环境变量文件,并在文件中为每个“键-值”对定义一个CMake变量。文件的内容如下所示:

\begin{lstlisting}[style=styleCMake]
KEY1="Value1"
KEY2="Value2"
\end{lstlisting}

本节中,将遵循chapter13/ex02\_envfile\_utility的示例。从cmake/envfile-utils.cmake开始:

\begin{lstlisting}[style=styleCMake]
include_guard(DIRECTORY)
function(read_environment_file ENVIRONMENT_FILE_NAME)
	file(STRINGS ${ENVIRONMENT_FILE_NAME} KVP_LIST ENCODING
		UTF-8)
	foreach(ENV_VAR_DECL IN LISTS KVP_LIST)
		string(STRIP ENV_VAR_DECL ${ENV_VAR_DECL})
		string(LENGTH ENV_VAR_DECL ENV_VAR_DECL_LEN)
		if(ENV_VAR_DECL_LEN EQUAL 0)
			continue()
		endif()
		string(SUBSTRING ${ENV_VAR_DECL} 0 1
			ENV_VAR_DECL_FC)
		if(ENV_VAR_DECL_FC STREQUAL "#")
			continue()
		endif()
		string(REPLACE "=" ";" ENV_VAR_SPLIT
		${ENV_VAR_DECL})
		list(GET ENV_VAR_SPLIT 0 ENV_VAR_NAME)
		list(GET ENV_VAR_SPLIT 1 ENV_VAR_VALUE)
		string(REPLACE "\"" "" ENV_VAR_VALUE
			${ENV_VAR_VALUE})
		set(${ENV_VAR_NAME} ${ENV_VAR_VALUE} PARENT_SCOPE)
	endforeach()
endfunction()
\end{lstlisting}

envfile-utils.cmake实用工具模块包含函数read\_environment\_file,以键值对列表的格式读取一个环境文件。这个函数将文件中的所有行读入KVP\_LIST变量,然后遍历所有行。每一行都用(=)等号标记分隔,然后等号的左侧作为变量名,而右侧作为变量值,将每个“键-值”对定义为CMake变量,空行和注释行将跳过。关于模块的使用,来看看chapter13/ex02\_envfile\_utility/CMakeLists.txt:

\begin{lstlisting}[style=styleCMake]
# Add cmake folder to the module path, so subsequent 
# include() calls
# can directly include modules under cmake/ folder by
# specifying the name only.
set(CMAKE_MODULE_PATH ${CMAKE_MODULE_PATH}
	${PROJECT_SOURCE_DIR}/cmake/)
add_subdirectory(test-executable)
\end{lstlisting}

可能已经注意到,cmake文件夹现在添加到CMAKE\_MODULE\_PATH变量中,该变量是\texttt{include(…)}指令将搜索的路径集合,默认情况下是空。所以现在可以直接在当前和子CMakeLists.txt中使用名称包含envfile-utils模块。最后,看看chapter13/ex02\_envfile\_utility/test-executable/CMakeLists.txt:

\begin{lstlisting}[style=styleCMake]
# ....
# Include the module by name
include(envfile-utils)
read_environment_file("${PROJECT_SOURCE_DIR}/variables.env")
add_executable(ch13_ex02_envfile_utility_test)
target_sources(ch13_ex02_envfile_utility_test
	PRIVATE test.cpp)
target_compile_features(ch13_ex02_envfile_utility_test
	PRIVATE cxx_std_11)
target_compile_definitions(ch13_ex02_envfile_utility_test
	PRIVATE PROJECT_VERSION="${TEST_PROJECT_VERSION}"
		PROJECT_AUTHOR="${TEST_PROJECT_AUTHOR}")
\end{lstlisting}

可以按名称包含envfile-utils环境文件读取器模块,因为包含envfile-utils.cmake的文件夹已经添加到CMAKE\_MODULE\_PATH变量中。read\_environment\_file()函数用来读取同文件夹下的variables.env文件,包含以下键值对:

\begin{lstlisting}[style=styleCMake]
# This file contains some metadata about the project
TEST_PROJECT_VERSION="1.0.2"
TEST_PROJECT_AUTHOR="CBP Authors"
\end{lstlisting}

因此,使用read\_environment\_file()之后,期望TEST\_PROJECT\_VERSION和TEST\_PROJECT\_AUTHOR变量在当前的CMake作用域中定义,并在文件中指定各自的值。为验证这一点,定义了一个名为ch13\_ex02\_envfile\_utility\_test的可执行目标,并将TEST\_PROJECT\_VERSION和TEST\_PROJECT\_AUTHOR变量作为宏定义传递给目标。最后,目标的源文件test.cpp将TEST\_PROJECT\_VERSION和TEST\_PROJECT\_AUTHOR宏定义打印到控制台:

\begin{lstlisting}[style=styleCXX]
#include <cstdio>
int main(void) {
	std::printf("Version '%s', author '%s'\n",
		TEST_PROJECT_VERSION, TEST_PROJECT_AUTHOR);
}
\end{lstlisting}

好了,让我们编译并运行应用程序:

\begin{tcblisting}{commandshell={}}
cd chapter13/ex02_envfile_utility
cmake -S ./ -B ./build
cmake --build build
./build/test-executable/ch13_ex02_envfile_utility_test
# Will output: Version '1.0.2', author 'CBP Authors'
\end{tcblisting}

我们已经成功地从源树中读取了键值对格式的文件,将每个键值对定义为CMake变量,然后将这些变量作为宏定义公开给应用程序。

尽管编写CMake模块非常简单,但这里有一些建议:

\begin{itemize}
\item 
为函数/宏使用唯一的名称。

\item 
为所有模块函数/宏使用共同的前缀。

\item 
避免对非函数作用域变量使用常量名。

\item 
为模块使用\texttt{include\_guard()}。

\item 
若模块需要打印消息,请为模块提供静默模式。

\item 
不要暴露模块的内部。

\item 
为简单的命令包装器使用宏,其他的都使用函数。
\end{itemize}

接下来,我们将探讨在项目之间共享CMake模块的方法。

\subsubsubsection{13.4.2\hspace{0.2cm}项目间共享CMake模块的建议}

共享CMake模块的推荐方法是为CMake模块维护一个单独的项目,然后将该项目作为外部资源合并,可以直接通过Git子模块/子树或CMake的FetchContent/ExternalProject。这样,所有可重用的CMake实用程序都可以维护在单个项目下,并可以传播到所有下游项目。将CMake模块放入在线Git托管平台(如GitHub或GitLab)的存储库中,将使大多数人使用该模块更加方便。因为CMake支持直接从Git获取内容,所以使用共享库将非常简单。

为了演示如何使用外部CMake模块项目,我们将使用名为Hadouken的开源CMake实用模块项目。该项目可从\url{https://github.com/mustafakemalgilor/hadouken}访问。该项目包含用于工具集成、目标创建和特性检查的CMake实用工具模块。

这里,使用chapter13/ex03\_hadouken的例子。用这个例子获取Hadouken,然后使用Hadouken项目的目标创建助手实用程序。先来看看CMakeLists.txt:

\begin{lstlisting}[style=styleCMake]
cmake_minimum_required(VERSION 3.21)
project(
	ch13_ex03_hadouken
	VERSION 1.0
	DESCRIPTION
		"Chapter 13 Example 03, external CMake modules (hadouken) example"
	LANGUAGES CXX)
include(FetchContent)
# Declare hadouken dependency details.
FetchContent_Declare(hadouken
	GIT_REPOSITORY https://github.com/mustafakemalgilor/hadouken.git
	GIT_TAG 7d0447fcadf8e93d25f242b9bb251ecbcf67f8cb
	SOURCE_DIR "${CMAKE_CURRENT_LIST_DIR}/.hadouken"
)
FetchContent_MakeAvailable(hadouken)
set(CMAKE_MODULE_PATH ${CMAKE_MODULE_PATH}
	${PROJECT_SOURCE_DIR}/.hadouken/cmake/modules/)
include(misc/Log)
include(misc/Utility)
include(core/MakeCompilationUnit)
include(core/MakeTarget)
# Create an executable target by using Hadouken's
# make_target() utility function
make_target(TYPE EXECUTABLE)
\end{lstlisting}

前面的例子中,使用了\texttt{FetchContent\_Declare}和\texttt{FetchContent\_MakeAvailable}将Hadouken检索到项目中。然后,Hadouken项目的模块目录添加到CMAKE\_MODULE\_PATH中,通过\texttt{include(…)}指令使用Hadouken项目的CMake实用模块,从而包含了Log、Utility、MakeCompilationUnit和MakeTarget模块。最后,使用make\_target()函数确保我们可以使用外部CMake模块项目的函数。

make\_target(…)函数是由Hadouken项目的核心/MakeTarget模块提供的,是\texttt{add\_executable}和\texttt{add\_library}指令的包装函数。make\_target(TYPE EXECUTABLE)将包括src/文件夹下的所有源文件,并通过\texttt{add\_executable(…)}创建一个可执行目标。让我们来配置和构建项目,看看情况是否如此:

\begin{tcblisting}{commandshell={}}
cd chapter13/ex03_hadouken
cmake -S ./ -B build/
cmake --build build
\end{tcblisting}

输出应该如下所示:

\begin{tcblisting}{commandshell={}}
[ 50%] Building CXX object
  CMakeFiles/ch13_ex03_hadouken.dir/src/main.cpp.o
[100%] Linking CXX executable ch13_ex03_hadouken
[100%] Built target ch13_ex03_hadouken
\end{tcblisting}

定义了目标ch13\_ex03\_hadouken,并将源文件main.cpp作为源文件包含在目标中。证实了我们可以在CMake代码中使用外部CMake模块项目。

接下来,我们将总结本章学到的知识,以及下一章将学到的知识。











