
为项目编写构建系统代码不是一件容易的事。项目维护者和开发人员花费大量精力编写CMake代码,以配置编译器标志、项目构建变量、第三方库和工具集成。处理多个CMake项目时,从头开始为项目无关的细节编写CMake代码可能会带来很大的负担。为项目编写的配置上述细节的大部分CMake代码可以在项目之间重用。考虑到这一点,开发一种策略使CMake代码可以友好的重用会很有帮助。解决这个问题的直接方法是将CMake代码视为常规代码,并应用一些最基本的编码原则:不要重复自己(DRY)原则和单责任原则(SRP)。

若考虑到可重用性,CMake代码可以很容易地重用。实现基本的可重用非常简单:将CMake代码分离为模块和函数,使CMake代码可重用的方法与使软件代码可重用没有什么不同。CMake本身毕竟是一种脚本语言,可以很自然地将CMake代码视为常规代码,并在处理它时应用软件设计原则。与任何函数式脚本语言一样,CMake具有以下重用性的基本能力:

\begin{itemize}
\item 
能够包含其他CMake文件

\item 
函数/宏

\item 
可移植性
\end{itemize}

本章中,我们将学习为一个项目编写CMake代码的方法,并考虑到可重用性,以及在CMake项目中重用CMake代码。还将讨论版本控制和在项目之间共享通用CMake代码的方法。

将讨论以下主题:

\begin{itemize}
\item 
了解CMake模块

\item 
模块的基本构建块——函数和宏

\item 
编写一个CMake模块
\end{itemize}
