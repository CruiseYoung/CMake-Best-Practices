
In this chapter, you learned about some concepts and strategies for migrating projects of various sizes to CMake. The effort and actual work to be done when migrating projects to CMake will very much depend on the individual setup of a project. However, with the approaches described here, choosing the right strategy will hopefully be easier. Changing the build processes and developer workflow is often disruptive, so you have to carefully consider whether the effort is worth it. Nevertheless, switching a project to CMake will open up the possibilities of all the features and practices for building quality software, as described in this book. Additionally, having a clean and well-maintained build system to work with will allow developers to focus on their main task, which is writing code and shipping software.

This brings us to the final chapter of this book, which is about getting access to the CMake community, finding further reading material, and contributing to CMake itself.