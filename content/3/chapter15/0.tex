虽然CMake正在成为C++和C项目事实上的工业标准,但仍然有一些项目(有时是大型项目)使用不同的构建系统。当然,只要它符合需求,就没有什么错。然而,有时候,无论出于某些原因,可能希望切换到CMake。例如,也许软件在不同的IDE或不同的平台上构建,或者依赖管理变得很麻烦。另一种情况是,仓库结构从大型单体仓库变更为每个库项目的分布式仓库。无论什么原因,迁移到CMake可能是一个挑战,特别是对于大型项目,但结果可能是值得的。

虽然一次性转换项目是首选的方式,但通常情况下,非技术需求可能无法做到这一点。例如,在迁移过程中可能仍然需要在某些部分进行开发,或者由于各种超出团队控制范围的需求,项目的某些部分无法从一开始就迁移。

因此,我们通常需要循序渐进的方法。更改构建系统很可能会影响CI/CD流程,因此也应该考虑这一点。本章中,我们将介绍一些高层策略,了解如何将项目逐步迁移到CMake。但请注意,具体的迁移路径非常依赖于个别情况。例如,从一个基于makefile的项目迁移到一个单独的仓库,和从一个基于gradle的跨多个仓库的项目迁移的效果是不同的。

更改构建系统,甚至可能更改项目结构,对所有参与的人来说可能具有破坏性的,因为他们将习惯于使用现有的结构和构建系统。因此,不应该轻率地决定切换到构建系统,而应该只在由显著好处的情况下才这么做。

虽然本章主要关注迁移项目的CMake方面,但迁移并不以切换构建系统为目标,而是有其他主要目标,例如简化项目结构或减少项目各部分之间的耦合,以使它们更容易独立维护。在谈到这些好处时,它们不一定是纯技术上的好处,例如可以更好地并行化构建,从而获得更快的构建速度。好处还可能更多地来自“社交”方面,例如:拥有标准化的、众所周知的软件开发方法将减少新开发人员的时间。

本章中,我们将讨论以下内容:

\begin{itemize}
\item 
高级迁移策略

\item 
迁移小项目

\item 
将大型项目迁移到CMake
\end{itemize}

本章中,将介绍一些从构建系统迁移到CMake的高级概念。迁移小型项目可能非常简单,而大型、复杂的项目则需要预先进行更多的规划。本章结束时,您将对将不同规模的项目迁移到CMake的不同策略有一个很好的了解。此外,我们将提供一些关于迁移时要检查什么的提示,迁移的大致分步指南,以及如何与遗留构建系统交互。