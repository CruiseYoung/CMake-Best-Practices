Building a software project is only half the story. The other half is about delivering and presenting the software to your consumers. Consumers are the biggest stakeholders of any project, even if you are writing a hobby project for yourself. These consumers may have a variety of experiences and purposes. They might be developers, package maintainers, power users, or the average Joes. It is important to understand their use cases, scenarios, and requirements. Since the software is mostly abstract, let's assume that your project is baked beans instead – it may be delicious and may smell good in the factory, but improper packaging will reduce its shelf life, making it hard to transport or consume. This will make your product less likely to be desired by the consumers. Even though your product is wonderful at the start, the consumer will not notice it since their experience with your product is bad due to bad packaging. Therefore, it is important to get these things right from the start. Remember, happy consumers will bring value to a product.

CMake has good internal support and tooling for making installing and packaging easy. The good side of this is that CMake leverages the existing project code to do such things. Thus, making a project installable or packaging a project does not result in heavy maintenance costs. In this chapter, we're going to learn how to leverage CMake's existing abilities regarding installing and packaging for deployments.

In this chapter, we will cover the following topics:

\begin{itemize}
\item 
Making CMake targets installable

\item 
Supplying configuration information for others using your project

\item 
Creating an installable package with CPack
\end{itemize}
