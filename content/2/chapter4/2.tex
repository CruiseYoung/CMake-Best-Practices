项目中支持部署的方法是可安装,这里的"可安装"并不是指安装预先制作好的软件包,而是用户必须获取项目的源码,并从头开始构建。可安装的项目需要额外的构建系统,用于在系统上安装运行时或开发构件。构建系统将在这里执行安装操作,因为这里使用CMake来生成构建系统文件,所以CMake必须生成相关的安装代码。本节中,将了解如何来写这部分CMake代码。

\subsubsubsection{4.2.1\hspace{0.2cm}install()指令}

\texttt{install(…)}指令是一个内置的CMake命令,允许生成用于安装目标、文件、目录等的构建系统说明。CMake不会生成安装指令,除非明确地说明。因此,安装总是在开发者的控制中。来看下它的基本用法。

\hspace*{\fill} \\ %插入空行
\noindent
\textbf{安装CMake目标}

要使CMake目标可安装,必须用至少一个参数指定TARGETS参数。指令的签名如下所示
:

\begin{lstlisting}[style=styleCMake]
install(TARGETS <target>... [...])
\end{lstlisting}

TARGETS参数表示\texttt{install}可以接受一组CMake目标,以便为其生成安装代码,只安装目标的输出构件。最常见的目标输出工件定义如下:

\begin{itemize}
\item 
ARCHIVE (静态库、DLL导入库和链接器导入文件):

\begin{itemize}
\item 
除了在macOS中标记为FRAMEWORK的目标
\end{itemize}

\item 
LIBRARY (动态库):

\begin{itemize}
\item 
除了在macOS中标记为FRAMEWORK的目标

\item 
除了dll (Windows)
\end{itemize}

\item 
RUNTIME (可执行文件和dll):

\begin{itemize}
\item 
除了在macOS中标记为MACOSX\_BUNDLE的目标
\end{itemize}
\end{itemize}

使目标可安装之后,CMake将生成必要的安装代码,以安装输出的构件,这里先将可执行目标安装在一起。要查看\texttt{install(…)}指令的运行情况,可以看看第4章示例1的CMakeLists.txt文件,可以在chapter4/ex01\_executable文件夹中找到:

\begin{lstlisting}[style=styleCMake]
add_executable(ch4_ex01_executable)
target_sources(ch4_ex01_executable src/main.cpp)
target_compile_features(ch4_ex01_executable PRIVATE cxx_std_11)
install(TARGETS ch4_ex01_executable)
\end{lstlisting}

前面的代码中,定义了一个名为ch4\_ex01\_executable的可执行目标,并在随后的两行中填充了它的属性。最后一行\texttt{install(…)}是我们感兴趣的,是ch4\_ex01\_executable所需的安装代码。

要检查ch4\_ex01\_executable是否可以安装,可以构建项目并通过CLI进行安装:

\begin{tcblisting}{commandshell={}}
cmake –S . -B ./build
cmake --build ./build
cmake --install ./build --prefix /tmp/install-test
\end{tcblisting}

\begin{tcolorbox}[colback=webgreen!5!white,colframe=webgreen!75!black,title=Note]
不必为cmake -{}-install指定-{}-prefix参数,也可以使用CMAKE\_INSTALL\_PREFIX来修改非默认的安装路径。

当使用CMake和多配置生成器(如Ninja multi-config和Visual Studio)时,请为\texttt{cmake -{}-build}和\texttt{cmake -{}-install}指定-{}-config参数:

\begin{tcblisting}{commandshell={}}
# For multi-config generators:
cmake --build ./build --config Debug
cmake --install ./build --prefix /tmp/install-test
--config Debug
\end{tcblisting}
\end{tcolorbox}

看一下\texttt{cmake -{}-install}做了什么:

\begin{tcblisting}{commandshell={}}
-- Install configuration: ""
-- Installing: /tmp/install-test/lib/libch2.framework.component1.a
-- Installing: /tmp/install-test/lib/libch2.framework.component2.so
-- Installing: /tmp/install-test/bin/ch2.driver_application
-- Set runtime path of "/tmp/install-test/bin/
    ch2.driver_application" to ""
-- Installing: /tmp/install-test/bin/ch4_ex01_executable
\end{tcblisting}

前面输出的最后一行中,可以看到ch4\_ex01\_executable目标的输出构件——安装了ch4\_ex01\_executable二进制文件。由于ch4\_ex01\_executable目标拥有的唯一输出工件,所以该目标确实已经可安装。

注意ch4\_ex01\_executable没有直接安装在/tmp/installtest目录中,\texttt{install}将它放在bin子目录中,这是因为CMake知道什么样的工件应该放在哪里。传统的UNIX系统中,二进制文件存放在/usr/bin中,而库文件存放在/usr/lib中,CMake知道\texttt{add\_executable()}会生成一个可执行的二进制文件,所以将其放入bin子目录中。这些目录的默认值由CMake提供,具体取决于目标类型。提供默认安装路径信息的CMake模块称为GNUInstallDirs模块。GNUInstallDirs模块定义了各种CMAKE\_INSTALL\_的路径。默认安装目录如下表所示:

\begin{table}[H]
	\centering
	\begin{tabular}{|l|l|l|}
		\hline
		\textbf{目标类型} & \textbf{GNUInstallDirs变量} & \textbf{默认位置} \\ \hline
		RUNTIME              & \$\{CMAKE\_INSTALL\_BINDIR\}     & bin                       \\ \hline
		LIBRARY              & \$\{CMAKE\_INSTALL\_LIBDIR\}     & lib                       \\ \hline
		ARCHIVE              & \$\{CMAKE\_INSTALL\_LIBDIR\}     & lib                       \\ \hline
		PRIVATE\_HEADER      & \$\{CMAKE\_INSTALL\_INCLUDEDIR\} & include                   \\ \hline
		PUBLIC\_HEADER       & \$\{CMAKE\_INSTALL\_INCLUDEDIR\} & include                   \\ \hline
	\end{tabular}
\end{table}

为了覆盖内置默认值,在\texttt{install(…)}指令中需要使用<TARGET\_TYPE> DESTINATION参数。这里,我们尝试将默认的运行时安装目录改为qbin,而不是bin。只需要对\texttt{install(…)}指令做一个小修改:

\begin{lstlisting}[style=styleCMake]
# …
install(TARGETS ch4_ex01_executable
	RUNTIME DESTINATION qbin
)
\end{lstlisting}

进行修改后,可以重新运行配置、构建和安装命令。可以通过\texttt{cmake -{}-install}检查输出来确认运行时目标已经更改。与第一次不同,可以观察到ch4\_ex01\_executable二进制文件放入了qbin,而不是默认的(bin)目录中:

\begin{tcblisting}{commandshell={}}
# ...
-- Installing: /tmp/install-test/qbin/ch4_ex01_executable
\end{tcblisting}

现在,来看另一个例子,这次将安装STATIC库。回顾一下第4章示例2的CMakeLists.txt文件,可以在chapter4/ex02\_static文件夹中找到。由于空格的原因,注释和\texttt{project(…)}省略:

\begin{lstlisting}[style=styleCMake]
add_library(ch4_ex02_static STATIC)
target_sources(ch4_ex02_static PRIVATE src/lib.cpp)
target_include_directories(ch4_ex02_static PUBLIC include)
target_compile_features(ch4_ex02_static PRIVATE cxx_std_11)
include(GNUInstallDirs)
install(TARGETS ch4_ex02_static)
install (
	DIRECTORY include/
	DESTINATION "${CMAKE_INSTALL_INCLUDEDIR}"
)
\end{lstlisting}

这与我们前面的示例有一点不同。首先,会有多一个DIRECTORY参数,这是使静态库的头文件可安装。这样做的原因是CMake不会安装任何非输出工件,而STATIC库目标只生成一个二进制文件作为输出工件。头文件不是输出工件,应该单独安装。

\begin{tcolorbox}[colback=webgreen!5!white,colframe=webgreen!75!black,title=Note]
DIRECTORY参数中末尾的斜杠会导致CMake复制文件夹的内容,而不是按名称复制文件夹。CMake处理尾随斜杠的方式与Linux rsync命令相同。
\end{tcolorbox}

\subsubsubsection{4.2.2\hspace{0.2cm}安装文件和目录}

正如前一节中看到的,安装的东西并不总是目标输出构件的一部分。它们可能是目标的运行时依赖项,例如图片、源文件、脚本和配置文件。CMake提供了\texttt{install(FILES…)}和\texttt{install(DIRECTORY…)}指令来安装任何特定的文件或目录。先从安装文件开始吧!

\hspace*{\fill} \\ %插入空行
\noindent
\textbf{安装文件}
 
\texttt{install(FILES…)}指令接受一个或多个文件作为参数,还需要TYPE或DESTINATION参数,这两个参数用于确定指定文件的目标目录。TYPE用于指示哪些文件将使用该文件类型的默认路径作为安装目录,可以通过设置相关的GNUInstallDirs变量来重写默认值。下表显示了有效的TYPE值,以及它们的目录映射:

\begin{table}[H]
	\centering
	\begin{tabular}{|l|l|l|}
		\hline
		\textbf{类型} & \textbf{GUNInstallDirs变量}    & \textbf{默认值}                 \\ \hline
		BIN           & \$\{CMAKE\_INSTALL\_BINDIR\}        & bin                                       \\ \hline
		LIB           & \$\{CMAKE\_INSTALL\_SBINDIR\}       & sbin                                      \\ \hline
		LIB           & \$\{CMAKE\_INSTALL\_LIBDIR\}        & lib                                       \\ \hline
		INCLUDE       & \$\{CMAKE\_INSTALL\_INCLUDEDIR\}    & include                                   \\ \hline
		SYSCONF       & \$\{CMAKE\_INSTALL\_SYSCONFDIR\}    & etc                                       \\ \hline
		SHAREDSTATE   & \$\{CMAKE\_INSTALL\_SHARESTATEDIR\} & com                                       \\ \hline
		LOCALSTATE    & \$\{CMAKE\_INSTALL\_LOCALSTATEDIR\} & var                                       \\ \hline
		RUNSTATE & \$\{CMAKE\_INSTALL\_RUNSTATEDIR\} & \textless{}LOCALSTATE dir\textgreater{}/run  \\ \hline
		DATA          & \$\{CMAKE\_INSTALL\_DATADIR\}       & \textless{}DATAROOT dir\textgreater{}     \\ \hline
		INFO     & \$\{CMAKE\_INSTALL\_INFODIR\}     & \textless{}DATAROOT dir\textgreater{}/info   \\ \hline
		LOCALE   & \$\{CMAKE\_INSTALL\_LOCALEDIR\}   & \textless{}DATAROOT dir\textgreater{}/locale \\ \hline
		MAN           & \$\{CMAKE\_INSTALL\_MANDIR\}        & \textless{}DATAROOT dir\textgreater{}/man \\ \hline
		DOC           & \$\{CMAKE\_INSTALL\_DOCDIR\}        & \textless{}DATAROOT dir\textgreater{}/doc \\ \hline
	\end{tabular}
\end{table}

若不想使用TYPE,可以使用DESTINATION。可以通过\texttt{install(…)}为指定文件提供自定义目标。

另一种形式的\texttt{install(FILES...)}是\texttt{install(PROGRAMS...)},这与\texttt{install(FILES...)}相同,但需要为安装文件设置OWNER\_EXECUTE,GROUP\_EXECUTE和WORLD\_EXECUTE权限。这对最终用户执行的二进制文件或脚本文件是有意义的。

为了理解\texttt{install(FILES|PROGRAMS…)},来看一个示例,第4章的例3(chapter\_4/ex03\_file)。其包含三个文件:chapter4\_greeter\_content、chapter4\_greeter.py和CMakeLists.txt。首先,看看CMakeLists.txt:

\begin{lstlisting}[style=styleCMake]
install(FILES "${CMAKE_CURRENT_LIST_DIR}/chapter4_greeter_content"
	DESTINATION "${CMAKE_INSTALL_BINDIR}")
	install(PROGRAMS "${CMAKE_CURRENT_LIST_DIR}/chapter4_greeter.py"
	DESTINATION "${CMAKE_INSTALL_BINDIR}" RENAME chapter4_greeter)
\end{lstlisting}

第一个\texttt{install(…)}中,CMake在当前的CMakeLists.txt(chapter4/ex03\_file)中安装了chapter4\_greeter\_content文件,在系统默认BIN目录下。第二个\texttt{install(…)}中,CMake在默认BIN目录中安装chapter4\_greeter.py,名称为chapter4\_greeter。

\begin{tcolorbox}[colback=webgreen!5!white,colframe=webgreen!75!black,title=Note]
RENAME参数只对单文件的\texttt{install(…)}有效。
\end{tcolorbox}

有了这些\texttt{install(…)}指令,CMake应该在\$\{CMAKE\_INSTALL\_PREFIX\}/bin目录下安装chapter4\_greeter.py和chapter4\_greeter\_content。通过CLI来构建和安装这个项目:

\begin{tcblisting}{commandshell={}}
cmake –S . -B ./build
cmake --build ./build
cmake --install ./build --prefix /tmp/install-test
\end{tcblisting}

看看\texttt{cmake -{}-install}做了什么:

\begin{tcblisting}{commandshell={}}
/* … */
-- Installing: /tmp/install-test/bin/chapter4_greeter_content
-- Installing: /tmp/install-test/bin/chapter4_greeter
\end{tcblisting}

上面的输出确认了CMake为chapter4\_greeter\_content和chapter4\_greeting.py文件生成了所需的安装代码。最后,因为使用了\texttt{PROGRAMS}参数来安装chapter4\_greeter,检查一下是否可以执行:

\begin{tcblisting}{commandshell={}}
15:01 $ /tmp/install-test/bin/chapter4_greeter
['Hello from installed file!']
\end{tcblisting}

至此,已经了解了\texttt{install(FILES|PROGRAMS…)}。接下来,就是安装目录。

\hspace*{\fill} \\ %插入空行
\noindent
\textbf{安装目录}

\texttt{install(DIRECTORY…)}指令用于安装目录,目录的结构将按原样复制到目标。目录既可以作为一个整体安装,也可以有选择地安装。先从最基本的目录安装示例开始:

\begin{lstlisting}[style=styleCMake]
install(DIRECTORY dir1 dir2 dir3 TYPE LOCALSTATE)
\end{lstlisting}

前面的示例将把dir1和dir2目录安装到\$\{CMAKE\_INSTALL\_PREFIX\}/var目录中,以及所有子文件夹和文件。有时,不能直接安装文件夹的全部内容。幸运的是,CMake允许安装命令根据通配符模式和正则表达式包含或排除目录内容。这次我们有选择地安装dir1、dir2和dir3:

\begin{lstlisting}[style=styleCMake]
include(GNUInstallDirs)
install(DIRECTORY dir1 DESTINATION ${CMAKE_INSTALL_
	LOCALSTATEDIR} FILES_MATCHING PATTERN "*.x")
install(DIRECTORY dir2 DESTINATION ${CMAKE_INSTALL_
	LOCALSTATEDIR} FILES_MATCHING PATTERN "*.hpp"
EXCLUDE PATTERN "*")
install(DIRECTORY dir3 DESTINATION ${CMAKE_INSTALL_
	LOCALSTATEDIR} PATTERN "bin" EXCLUDE)
\end{lstlisting}

前面的例子中,使用FILES\_MATCHING参数来定义文件选择的条件。FILES\_MATCHING后面可以跟PATTERN或REGEX参数。PATTERN可以定义通配符模式,而REGEX可以定义正则表达式。通常,这些表达式用于包含文件。若希望排除匹配条件的文件,可以将EXCLUDE参数附加到模式中。由于FILES\_MATCHING参数,这些过滤器没有应用于子目录名。我们还在最后一个\texttt{install(…)}指令中使用了PATTERN,并且没有添加FILES\_MATCHING前缀,这可以过滤子目录而不是文件。这将只安装dir1中扩展名为.x的文件,dir2中没有扩展名为.hpp的文件,以及dir3中除bin文件夹外的所有内容。这个例子可以在第4章找到,在chapter4/ex04\_directory文件夹中。让我们编译并安装,看看是否符合我们的预期:

\begin{tcblisting}{commandshell={}}
cmake –S . -B ./build
cmake –build ./build
cmake –install ./build –prefix /tmp/install-test
\end{tcblisting}

\texttt{cmake -{}-install}的输出应该是这样的:

\begin{tcblisting}{commandshell={}}
-- Installing: /tmp/install-test/var/dir1
-- Installing: /tmp/install-test/var/dir1/subdir
-- Installing: /tmp/install-test/var/dir1/subdir/asset5.x
-- Installing: /tmp/install-test/var/dir1/asset1.x
-- Installing: /tmp/install-test/var/dir2
-- Installing: /tmp/install-test/var/dir2/chapter4_hello.dat
-- Installing: /tmp/install-test/var/dir3
-- Installing: /tmp/install-test/var/dir3/asset4
\end{tcblisting}

\begin{tcolorbox}[colback=webgreen!5!white,colframe=webgreen!75!black,title=Note]
FILES\_MATCHING不能在PATTERN或REGEX之后使用,反之亦然。
\end{tcolorbox}

输出中,可以看到只有扩展名为.x的文件是从dir1中选取的,因为在第一个\texttt{install(…)}中使用FILES\_MATCHING PATTERN "*.x",导致asset2文件没有被安装。另外,dir2/chapter4\_hello.dat文件已经安装,并且跳过dir2/chapter4\_hello.hpp文件,这是因为在第二个\texttt{install(…)}中指定了FILES\_MATCHING PATTERN "*.hpp" EXCLUDE PATTERN "*"。最后,安装了dir3/asset4文件,并且完全跳过了dir3/bin目录,因为在最后一个\texttt{install(…)}中指定了PATTERN "bin" EXCLUDE参数。

通过\texttt{install(DIRECTORY…)},了解了\texttt{install(…)}的基本知识。接下来,继续看看\texttt{install(…)}的其他常见参数。

\subsubsubsection{4.2.3\hspace{0.2cm}install()的其他常见参数}

\texttt{install()}的第一个参数指示要安装什么,还有一些参数允许进行定制安装。让我们一起了解一下这些常见参数。

\hspace*{\fill} \\ %插入空行
\noindent
\textbf{DESTINATION}

这个参数允许\texttt{install(…)}指定安装目录,目录可以是相对路径,也可以是绝对路径。相对路径是相对于CMAKE\_INSTALL\_PREFIX的,建议使用相对路径使安装可重定位。另外,使用相对路径进行打包也很重要,因为cpack要求安装路径是相对路径。最好使用以相关GNUInstallDirs变量开始的路径,以便包维护人员在需要时覆盖安装目标。DESTINATION参数可以与TARGETS,FILES,IMPORTED\_RUNTIME\_ARTIFACTS,EXPORT和DIRECTORY安装类型一起使用。

\hspace*{\fill} \\ %插入空行
\noindent
\textbf{PERMISSIONS}

此参数允许在支持的平台上更改已安装文件的权限。可用权限为:OWNER\_READ、OWNER\_WRITE、OWNER\_EXECUTE、GROUP\_READ、GROUP\_WRITE、GROUP\_EXECUTE、WORLD\_READ、WORLD\_WRITE、WORLD\_EXECUTE、SETUID和SETGID。PERMISSIONS参数可以与TARGETS, FILES, IMPORTED\_RUNTIME\_ARTIFACTS, EXPORT和DIRECTORY安装类型一起使用。

\hspace*{\fill} \\ %插入空行
\noindent
\textbf{CONFIGURATIONS}

允许指定构建配置时限制参数的应用。

\hspace*{\fill} \\ %插入空行
\noindent
\textbf{OPTIONAL}

此参数使要安装的文件为可选文件,因此当文件不存在时,安装不会失败。可选参数可以与TARGETS, FILES, IMPORTED\_RUNTIME\_ARTIFACTS和DIRECTORY安装类型一起使用。

\hspace*{\fill} \\ %插入空行
本节中,学习了如何使目标、文件和目录可安装。下一节中,将学习如何生成配置信息,以便可以将CMake项目直接导入到另一个CMake项目中。











