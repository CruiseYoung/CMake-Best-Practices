Before you dive further into this chapter, it is recommended to take a look at Chapter
7, Seamless Integration of Code-Quality Tools with CMake. This chapter will follow the teaching by example approach. Thus, it is recommended to obtain this chapter's example content from \url{https://github.com/PacktPublishing/CMake-BestPractices/tree/main/chapter\_11}. All of the examples assume that you will be using the container provided by the \url{https://github.com/PacktPublishing/CMake-Best-Practices project}.

Note that libFuzzer is a part of the LLVM toolchain and requires LLVM to be installed and used as a toolchain. The bundled development environment has llvm-13 pre-installed. AFL++ is a third-party tool that can be obtained from \url{https://github.com/AFLplusplus/AFLplusplus} to follow examples regarding AFL++. The installation guide is available at \url{https://github.com/AFLplusplus/AFLplusplus/blob/stable/docs/INSTALL.md}. Note that some distros have a pre-built version of AFL++ on their repositories. You may find it convenient to perform the installation via a package manager if that is the case. 

Let's start by learning some basics about fuzzing.