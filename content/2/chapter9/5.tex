One of the main strengths of CMake is its versatility to build software using a variety of toolchains for a large number of platforms. The downside of this is that it sometimes can be hard for developers to find a working configuration for software. But by supplying CMake presets, containers, and sysroots, it often gets easier to get started with a CMake project.

In this chapter, we looked in detail into how to define CMake presets to define working configuration setups, along with creating build and test definitions. Then, we briefly covered how to create a Docker container and how to invoke CMake commands within, before closing the chapter with a brief look into sysroots and toolchain files. More about toolchains and sysroots will be covered in Chapter 12, Cross-Platform Compiling and Custom Toolchains.

In the next chapter, you will learn how to work with big, distributed projects as super builds. There, you will learn how to handle different versions and how to assemble projects from multiple repositories in a manageable way.