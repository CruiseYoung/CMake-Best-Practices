CMake的主要优势是它的多功能性,可以使用各种工具链为大量平台构建软件。这样的缺点是,开发人员有时很难找到软件的工作配置。但是通过提供CMake预设、容器和sysroot,可使CMake项目会变得更容易使用。

本章中,我们详细研究了如何定义CMake预设来定义工作配置设置,以及如何创建构建和测试定义。然后,简要介绍了如何创建Docker容器,以及如何在其中调用CMake命令。然后以简要介绍sysroot和工具链文件结束本章。关于工具链和sysroot的更多内容将在第12章中介绍。

下一章中,将学习如何使用大型分布式项目作为超级构建,将了解如何处理不同的版本,以及如何以可管理的方式使用多个库组装项目。