
大多数项目的规模和复杂性都会迅速增长,它们依赖于文件、库,甚至可能是在项目外部管理的程序。CMake提供了内置指令可用来进行查找。搜索和寻找东西的过程似乎相当简单,但仔细分析一下,有很多事情需要考虑。首先,必须处理在哪里查找文件的搜索顺序。然后,可能想要添加文件可能所在的其他位置。最后,必须考虑不同操作系统之间的差异。

在单个文件的更高抽象级别上,CMake能够查找定义目标、包括路径和包特定变量的整个包。更多细节请参考CMake项目部分中的库。

有五个\texttt{find\_…}指令,它们选项和行为非常相似:

\begin{itemize}
\item 
find\_file: 定位单个文件。

\item 
find\_path: 查找包含特定文件的目录。

\item 
find\_library: 查找库文件。

\item 
find\_program: 查找可执行程序。

\item 
find\_package: 查找完整的包。
\end{itemize}

所有这些命令都以类似的方式工作,但在查找位置方面有一些区别。特别是,\texttt{find\_package}不仅定位文件、查找包,还使得文件在CMake项目中更易于使用。本章中,在介绍如何查找包之前,先看一下比较简单的\texttt{find}函数。

\subsubsubsection{5.2.1\hspace{0.2cm}查找文件和路径}

要查找的最底层和最基本的是文件和路径。\texttt{find\_file}和\texttt{find\_path}具有相同的签名,其唯一区别是\texttt{find\_path}存储的是文件目录,而\texttt{find\_file}将存储完整的路径,包括文件名。\texttt{find\_file}的签名如下:

\begin{lstlisting}[style=styleCMake]
find_file (
	<VAR>
	name | NAMES name1 [name2 ...]
	[HINTS [path | ENV var]... ]
	[PATHS [path | ENV var]... ]
	[PATH_SUFFIXES suffix1 [suffix2 ...]]
	[DOC "cache documentation string"]
	[NO_CACHE]
	[REQUIRED]
	[NO_DEFAULT_PATH]
	[NO_PACKAGE_ROOT_PATH]
	[NO_CMAKE_PATH]
	[NO_CMAKE_ENVIRONMENT_PATH]
	[NO_SYSTEM_ENVIRONMENT_PATH]
	[NO_CMAKE_SYSTEM_PATH]
	[CMAKE_FIND_ROOT_PATH_BOTH |
	ONLY_CMAKE_FIND_ROOT_PATH |
	NO_CMAKE_FIND_ROOT_PATH]
	)
\end{lstlisting}

上面的命令搜索单个文件(文件名),或者搜索可能的文件名列表(使用\texttt{NAMES}选项)。生成的路径存储在作为<VAR>变量中。若找不到文件,变量将为\texttt{<VARIABLENAME>-NOTFOUND}。

若要搜索的文件名有变化,比如不同的大写字母或命名约定,可能包含版本号,那么可以使用名称列表。传递名称列表时,当找到第一个文件,搜索就会停止,所以应按首选方式对名称进行排序。

\begin{tcolorbox}[colback=blue!5!white,colframe=blue!75!black,title=搜索包含版本号的文件]
建议在搜索包含某种版本编号形式的文件名前,先搜索没有版本编号的文件名。这是为了使本地构建的文件优于操作系统安装的文件。
\end{tcolorbox}

\texttt{HINTS}和\texttt{PATHS}选项包含搜索文件的默认位置之外的其他位置。\texttt{PATH\_SUFFIXES}可以包含许多子目录,在每个位置的下面搜索这些子目录。

\texttt{find\_…}指令在已定义的位置和顺序中进行搜索。\texttt{NO\_...\_PATH}参数可以用来跳过相应的位置。下表显示了搜索位置的顺序和跳过位置的选项:

\begin{table}[H]
	\centering
	\begin{tabular}{|l|l|}
		\hline
		\textbf{位置}                     & \textbf{跳过的选项} \\ \hline
		包的根变量                & NO\_PACKAGE\_ROOT\_PATH                 \\ \hline
		特定于CMake的缓存变量        & NO\_CMAKE\_PATH                         \\ \hline
		特定于CMake的环境变量  & NO\_CMAKE\_ENVIRONMENT\_PATH            \\ \hline
		HINTS选项中的路径           &                                         \\ \hline
		系统环境变量 & NO\_SYSTEM\_ENVIRONMENT\_PATH           \\ \hline
		系统缓存变量       & NO\_CMAKE\_SYSTEM\_PATH                 \\ \hline
		PATHS选项中的路径           &                                         \\ \hline
	\end{tabular}
\end{table}

仔细地了解一下搜索顺序,以及不同位置的含义:

\begin{itemize}
\item 
包的根变量: 这只在\texttt{find\_file}作为\texttt{find\_package}的一部分时使用才有用。请参考CMake项目中的“使用第三方库”部分。

\item 
特定于CMake的缓存变量: 从macOS的CMAKE\_PREFIX\_PATH,CMAKE\_INCLUDE\_PATH和CMAKE\_FRAMEWORK\_PATH缓存变量派生的位置。一般设置CMAKE\_PREFIX\_PATH缓存变量优于其他两种类型,因为它用于所有\texttt{find\_}指令。前缀路径是位于常用文件结构(如bin、lib、include等)的搜索基点。CMAKE\_PREFIX\_PATH是一个路径列表,对于其中每一个,\texttt{find\_file}都会搜索/include或/include/\$\{CMAKE\_LIBRARY\_ARCHITECTURE\}(若变量已经设置)。一般CMake会自动设置变量,开发人员不应该更改它们。特定于架构的路径优先于通用路径。

\item 
CMAKE\_INCLUDE\_PATH和CMAKE\_FRAMEWORK\_PATH缓存变量,仅当标准目录结构不适用时才应使用。它们不向路径添加额外的include后缀。

\item 
通过将NO\_CMAKE\_PATH选项传递给命令,或者全局地将CMAKE\_FIND\_USE\_PATH变量设置为false,可以跳过搜索这些路径。

\item 
系统环境变量: 从CMAKE\_PREFIX\_PATH、CMAKE\_INCLUDE\_PATH和CMAKE\_FRAMEWORK\_PATH系统环境变量派生。这些变量的工作方式与缓存变量相同,但通常在CMake的外部设置。

\item 
注意,在Unix平台上,列表用冒号(:)而不是分号(;)分隔,以符合特定于平台的环境变量。

\item 
HINTS选项中的路径: 这些是手动指定的附加搜索位置。可以从属性值等其他值构造,也可以依赖于以前找到的文件或路径。

\item 
系统环境变量: INCLUDE和PATH环境变量都可以包含要搜索的目录列表。同样,在Unix平台上,列表是用冒号(:),而不是分号(;)分隔的。

\item 
Windows上,以更复杂的方式处理PATHS。对于其中每个,通过删除尾随的bin或sbin目录来提取基本路径。若设置了CMAKE\_LIBRARY\_ARCHITECTURE,则添加include/\$\{CMAKE\_LIBRARY\_ARCHITECTURE\}子目录作为每个路径的第一优先级。之后,搜索include(不带后缀)。这样,才会搜索原始路径(可能以bin或sbin结束,也可能不以bin或sbin结束)。传递NO\_SYSTEM\_ENVIRONMENT\_PATH变量或将CMAKE\_FIND\_USE\_CMAKE\_SYSTEM\_PATH变量设置为false将跳过环境变量中的位置。

\item 
假设PATH选项包含C:\verb|\|myfolder\verb|\|bin;C:\verb|\|yourfolder,和CMAKE\_LIBRARY\_ARCHITECTURE设置为x86\_64,搜索顺序如下:

\begin{enumerate}
\item 
C:\verb|\|myfolder\verb|\|include\verb|\|x86\_64

\item 
C:\verb|\|myfolder\verb|\|include\verb|\|

\item 
C:\verb|\|myfolder\verb|\|bin

\item 
C:\verb|\|yourfolder\verb|\|include\verb|\|x86\_64

\item 
C:\verb|\|yourfolder\verb|\|include\verb|\|

\item 
C:\verb|\|yourfolder\verb|\|
\end{enumerate}

\item 
系统缓存变量: CMAKE\_SYSTEM\_PREFIX\_PATH和CMAKE\_SYSTEM\_FRAMEWORK\_PATH变量类似于CMAKE特定的缓存变量。这些变量不应该由开发人员更改,而是在CMake设置平台工具链时进行配置。例外是,若提供了自己的工具链文件,例如使用sysroot或交叉编译。

\item 
除了NO\_CMAKE\_SYSTEM\_PATH选项,CMAKE\_FIND\_USE\_CMAKE\_SYSTEM\_PATH变量可以设置为false,以跳过由系统特定的缓存变量提供的位置的搜索。

\item 
PATHS选项中的路径: 与提示HINTS相同,这些是手动提供的附加搜索位置。尽管在技术上没有禁止,但按照惯例,PATHS变量应该是固定路径,并且不依赖于其他值。
\end{itemize}

若只搜索由提示或路径提供的位置,添加NO\_DEFAULT\_PATH选项将跳过其他所有位置。

有时,可能希望忽略用于搜索的特定路径。这种情况下,路径列表可以在CMAKE\_IGNORE\_PATH或CMAKE\_SYSTEM\_IGNORE\_PATH中指定。这两个变量在设计时都考虑到了交叉编译场景,很少在其他情况下使用。

\hspace*{\fill} \\ %插入空行
\noindent
\textbf{交叉编译时搜索文件}

交叉编译时,因为交叉编译工具链会收集到它们自包含的目录结构中,所以搜索文件的过程通常不同,而不与系统工具链混合。首先,需要在工具链的目录中查找文件。通过设置CMAKE\_FIND\_ROOT变量,可以将所有搜索的原点更改为一个新位置。

此外,CMAKE\_SYSROOT、CMAKE\_SYSROOT\_COMPILE和CMAKE\_SYSROOT\_LINK变量会影响搜索位置,但它们应该只在工具链文件中设置,而不是由项目本身设置。若常规搜索位置已经在sysroot或CMAKE\_FIND\_ROOT指定的位置下,它们将不会更改。任何以波浪号(~)开头并传递给\texttt{find\_}指令的路径都不会更改,以避免跳过用户主目录下的目录。

默认情况下,CMake首先在上一段中提供的位置中搜索,然后继续搜索主机系统。这种行为可以通过设置CMAKE\_FIND\_ROOT\_PATH\_MODE\_INCLUDE变量为BOTH、NEVER或ONLY进行全局性修改。或者,可以为\texttt{find\_file}设置CMAKE\_FIND\_ROOT\_PATH\_BOTH,ONLY\_CMAKE\_FIND\_ROOT\_PATH,或者NO\_CMAKE\_FIND\_ROOT\_PATH。

下表显示了在不同的搜索模式下,设置选项或变量时的搜索顺序:

\begin{table}[H]
	\centering
	\begin{tabular}{|l|l|l|}
		\hline
		\textbf{模式} & \textbf{选项}             & \textbf{搜索顺序}         \\ \hline
		Both &
		CMAKE\_FIND\_ROOT\_PATH\_BOTH &
		\begin{tabular}[c]{@{}l@{}}·CMAKE\_FIND\_ROOT\_PATH\\ ·CMAKE\_SYSROOT\_COMPILE\\ ·CMAKE\_SYSROOT\_LINK\\ ·CMAKE\_SYSROOT\\ ·常规搜索位置\end{tabular} \\ \hline
		NEVER         & NO\_CMAKE\_FIND\_ROOT\_PATH & ·常规搜索位置 \\ \hline
		ONLY &
		ONLY\_CMAKE\_FIND\_ROOT\_PATH &
		\begin{tabular}[c]{@{}l@{}}CMAKE\_FIND\_ROOT\_PATH\\ ·CMAKE\_SYSROOT\_COMPILE\\ ·CMAKE\_SYSROOT\_LINK\\ ·CMAKE\_SYSROOT\\ ·常规路径或CMAKE\_STAGINF\_PREFIX路径下   \end{tabular} \\ \hline
	\end{tabular}
\end{table}

CMAKE\_STAGING\_PREFIX变量用于为交叉编译提供安装路径,不应该通过在CMAKE\_SYSROOT变量上安装东西来改变它。设置交叉编译工具链将在第11章中详细介绍。

\subsubsubsection{5.2.2\hspace{0.2cm}查找应用}

查找可执行文件非常类似于查找文件和路径,\texttt{find\_program}指令与\texttt{find\_file}几乎具有相同的签名。此外,\texttt{find\_program}有NAMES\_PER\_DIR选项,表明每次搜索一个目录,并在每个目录中搜索所有提供的文件名,而不是在每个目录中搜索每个文件。

在Windows上,.exe和.com文件扩展名会自动添加到提供的文件名中,而非.bat或.cmd。

\texttt{find\_program}使用的缓存变量与\texttt{find\_file}使用的缓存变量略有不同:

\begin{itemize}
\item 
\texttt{find\_program}自动将bin和sbin添加到由CMAKE\_PREFIX\_PATH提供的搜索位置。

\item 
CMAKE\_LIBRARY\_ARCHITECTURE中的值会忽略,没有任何作用。

\item 
CMAKE\_PROGRAM\_PATH代替了CMAKE\_INCLUDE\_PATH。

\item 
CMAKE\_APPBUNDLE\_PATH代替了CMAKE\_FRAMEWORK\_PATH。

\item 
CMAKE\_FIND\_ROOT\_PATH\_MODE\_PROGRAM用于更改搜索程序的模式。
\end{itemize}

与其他\texttt{find}指令一样,若CMake无法找到程序,\texttt{find\_program}将变量的值设置为<varname>-NOTFOUND。这通常有助于确定是否应该启用,依赖于某个外部程序的自定义构建步骤。

\subsubsubsection{5.2.3\hspace{0.2cm}查找库}

查找库是查找文件的特殊情况,因此\texttt{find\_library}支持与\texttt{find\_file}相同的选项集。与\texttt{find\_program}类似,也有附加的NAMES\_PER\_DIR选项,该选项首先检查所有文件名,然后再移动到下一个目录。查找常规文件和查找库之间的区别在于,\texttt{find\_library}自动将特定于平台的命名约定应用于文件名。在Unix平台上,名称将以lib作为前缀,而在Windows上,将添加.dll或.lib扩展名。

同样,缓存变量与\texttt{find\_file}和\texttt{find\_program}中使用的变量略有不同:

\begin{itemize}
\item 
\texttt{find\_library}通过CMAKE\_PREFIX\_PATH将lib添加到搜索位置,其使用CMAKE\_LIBRARY\_PATH代替CMAKE\_INCLUDE\_PATH来查找库。CMAKE\_FRAMEWORK\_PATH的作用类似于\texttt{find\_file}。CMAKE\_LIBRARY\_ARCHITECTURE的工作原理与\texttt{find\_file}相同。

\item 
这是通过将各自的文件夹附加到搜索路径来完成的。\texttt{find\_library}与\texttt{find\_file}搜索方式相同,也是遍历PATH环境变量中的位置,在每个前缀后面追加了lib。另外,若设置了LIB环境变量,则使用LIB环境变量,而不是INCLUDE。

\item 
CMAKE\_FIND\_ROOT\_PATH\_MODE\_LIBRARY用于更改库的搜索模式。
\end{itemize}

CMake通常知道关于32位和64位搜索位置的约定,例如:为相同名称的不同库使用lib32和lib64文件夹的平台。该行为由FIND\_LIBRARY\_USE\_LIB[32|64|X32]\_PATHS变量控制,该变量控制应该首先搜索什么。此外,项目可以使用CMAKE\_FIND\_LIBRARY\_CUSTOM\_LIB\_SUFFIX变量定义自己的后缀,该变量将覆盖其他变量的行为。然而,这样做的需求非常少,修改CMakeLists.txt文件中的搜索顺序很快就会使项目难以维护,并严重影响系统间的可移植性。

\hspace*{\fill} \\ %插入空行
\noindent
\textbf{查找静态或动态库}

大多数情况下,简单地将库名称传递给CMake就足够了。而有时,必须重写该行为。这样做的原因是,在某些平台上,是否应该优先使用静态版本的库,而非动态版本的库,反之亦然。做到这一点的最佳方法是将\texttt{find\_library}拆分为两个调用,而不是试图在一个调用中实现这一点。若静态库与动态库位于不同的目录中,则健壮性更强:

\begin{lstlisting}[style=styleCMake]
find_library(MYSTUFF_LIBRARY libmystuff.a)
find_library(MYSTUFF_LIBRARY mystuff)
\end{lstlisting}

Windows上不能使用这种方法,因为dll的静态库和导入库具有相同的名称.lib后缀,因此不能通过名称进行区分。\texttt{find\_file}、\texttt{find\_path}、\texttt{find\_program}和\texttt{find\_library}在查找特定的文件时很方便。另一方面,依赖性发生在更高的级别上,这正是\texttt{find\_package}的用武之地。使用\texttt{find\_package},不需要首先搜索所有的include文件,然后是库文件,手动将它们添加到目标中,并最终解释所有特定于平台的行为。接下来让我们深入研究如何查找依赖项。



