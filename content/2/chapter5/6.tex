本章中,介绍了查找文件、库和程序的一般方法,以及查找CMake包更复杂的方法。学习了若无法通过提供自己的find模块自动找到导入的包定义,如何创建该包定义。我们研究了基于源代码的依赖关系与\texttt{ExternalProject}和\texttt{FetchContent},以及如何使用CMake构建非CMake项目。

另外,若觉得依赖管理方面变得更加太复杂,这里简要介绍了Conan和vcpkg这两个包管理器,它们与CMake集成得非常好。

依赖管理是一个很难讨论的话题,有时可能很乏味,但花时间用本章描述的技术正确地设置它是值得的。CMake的多功能性,及其查找依赖项的方法是其优点,也是其缺点。通过使用各种\texttt{find\_}指令、\texttt{FetchContent}、\texttt{ExternalProject},或者将可用的包管理器与CMake集成,可以将依赖项集成到项目中。然而,有这么多的方法可以选择,找到最好的会很困难,但我们还是建议尽可能使用\texttt{find\_package}。CMake越受欢迎,其他项目就越有可能无缝集成。

在下一章中,将学习如何为代码自动生成和打包文档。