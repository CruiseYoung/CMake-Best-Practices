
将依赖项添加到项目中最简单的方法是使用apt-get、brew或Chocolatey进行安装。安装的缺点是,可能会使用许多不同版本的库污染系统,并且所寻找的版本可能根本不可用。若正在处理多个项目,同时对依赖关系有不同的需求,那么这一点尤其重要。通常,开发人员会在本地为每个项目下载依赖项,以便每个项目能够独立工作。处理依赖关系的一种方法是使用包管理器,例如Conan或vcpkg。

在依赖性管理方面,使用专用的包管理器有许多优点。处理C++依赖的两个比较流行的是Conan和vcpkg,都可以处理复杂的构建系统。想要完全了解它们需要阅读相关的书籍,所以这里只介绍使用它们所需的基本知识。本书中,我们将着重于使用CMake项目中已经提供的包,而不是自己创建的包。

\subsubsubsection{5.4.1\hspace{0.2cm}获取依赖——Conan}

过去的几年里,Conan包管理器获得了很大的普及,主要是因为它与CMake集成得非常好。Conan是一个分布式包管理器,构建在客户机/服务器架构上。所以,本地客户端可以获取或上传包到一个或多个远程服务器。

Conan最强大的特性是可以为多个平台、配置和版本创建和管理二进制包。创建包时,使用conanfile.py文件描述它们,该文件列出了所有依赖项、源和构建指令。

使用Conan客户机构建包并将其上传到远程服务器。这还有一个好处,若找不到适合本地配置的二进制包,可以使用源码在本地构建。

用CMake使用Conan的方法是使用CMake本身的Conan。若不想这样做,可以在外部使用Conan,但建议在使用Conan之前使用\texttt{find\_program}检查Conan程序是否存在。

为了直接从CMake中使用Conan,Conan提供了一个CMake包装程序供下载。下面的例子下载了conan-cmake包装器,然后从ConanCenter中提取fmt格式化库作为项目中的常规库:

\begin{lstlisting}[style=styleCMake]
if(NOT EXISTS "${CMAKE_CURRENT_BINARY_DIR}/conan.cmake")
	message(STATUS "Downloading conan.cmake from
		https://github.com/conan-io/cmake-conan")
	file( DOWNLOAD
		"https://raw.githubusercontent.com/conan-io/
			cmakeconan/0.17.0/conan.cmake"
		"${CMAKE_CURRENT_BINARY_DIR}/conan.cmake"
		EXPECTED_HASH
		SHA256=3bef79da16c2e031dc429e1dac87a08b9226418b300ce00
			4cc125a82687baeef
		STATUS download_status
	)
	
	if(NOT download_status MATCHES "^0;")
		message(FATAL_ERROR
			"Downloading conan.cmake failed with ${download_status}")
	endif()
endif()

include(${CMAKE_CURRENT_BINARY_DIR}/conan.cmake)

conan_cmake_autodetect(CONAN_SETTINGS)

conan_cmake_configure(REQUIRES fmt/6.1.2
	GENERATORS cmake_find_package_multi)
conan_cmake_install(PATH_OR_REFERENCE .
	BUILD missing
	SETTINGS ${CONAN_SETTINGS}
)

list(APPEND CMAKE_PREFIX_PATH ${CMAKE_CURRENT_BINARY_DIR})
find_package(fmt 6.1 REQUIRED)
add_executable(conan_example)
target_link_libraries(conan_example PRIVATE fmt::fmt)
\end{lstlisting}

CMake代码做了以下事情:

\begin{enumerate}
\item 
若还没有下载cmake-conan,conan.cmake文件将会下载到当前二进制目录下。

\item 
接下来,使Conan函数可用。

\item 
当包含Conan,就可以从当前的CMake配置中检测设置,如编译器、平台等,并将它们存储在CONAN\_SETTING变量中。

\item 
conan\_cmake\_configure函数定义了fmt,并为conan设置了生成器,这样就可以使用\texttt{find\_package}来包含依赖项。这将生成一个conanfile.txt文件,其中包含当前构建目录中Conan的必要指令。

\item 
最后,conan\_cmake\_install会安装相应的依赖项。

\item 
PATH\_OR\_REFERENCE会指出依赖关系的定义所在的位置。该命令的运行位置与conan\_cmake\_configure所处位置相同,因此将搜索相同的目录。若BUILD缺失,则这些包不能作为二进制文件从远程服务器获得,那就需要在本地进行构建。

\item 
SETTINGS将检索到设置传递给Conan。

\item 
由于生成的查找模块位于当前二进制目录中,所以必须添加到CMAKE\_MODULE\_PATH中。

\item 
完成下载后,可以使用\texttt{find\_package}来包含依赖项,然后像往常一样添加到现有目标中。
\end{enumerate}

还可以直接提供conanfile.txt文件:

\begin{lstlisting}[style=styleCMake]
[requires]
fmt/6.1.2
[generators]
cmake_find_package
\end{lstlisting}

不运行conan\_cmake\_configure和conan\_cmake\_install,而是通过conan\_cmake\_run调用conan。调整前面的示例以使用conanfile.txt文件将类似于以下内容(下载conan.cmake文件保持不变):

\begin{lstlisting}[style=styleCMake]
include(${CMAKE_CURRENT_BINARY_DIR}/conan.cmake)
conan_cmake_autodetect(CONAN_SETTINGS)
conan_cmake_run(CONANFILE ${CMAKE_CURRENT_LIST_DIR}/conanfile.txt
	BASIC_SETUP
	BUILD missing
	SETTINGS ${CONAN_SETTINGS})
list(APPEND CMAKE_PREFIX_PATH ${CMAKE_CURRENT_BINARY_DIR})
find_package(fmt 6.1 REQUIRED)
add_executable(conan_conanfile_example)
target_link_libraries(conan_conanfile_example PRIVATE fmt::fmt)
\end{lstlisting}

此设置与前面的示例类似,不同之处是Conan指向conanfile.txt文件。BASIC\_SETUP会告诉Conan自动创建必要的CMake变量,conan\_cmake\_run也可以用来运行所有conan命令。

当然,Conan也在CMake外部使用。有些人认为这是一种更简洁的方法,因为Conan和CMake之间的关注点是分离的,但维护起来会很繁琐。关于依赖项的信息必须在两个地方进行跟踪,而且构建配置,例如:编译器、libc版本、平台等,不仅必须为CMake配置,也必须为Conan配置。

完整的Conan文档可参阅\url{https://docs.conan.io/en/latest/}。

\subsubsubsection{5.4.2\hspace{0.2cm}使用vcpkg进行依赖管理}

另一个流行的开源包管理器是微软的vcpkg,工作方式类似于Conan,使用客户机/服务器架构。最初构建它是为了与Visual Studio编译器环境一起工作,后来添加了CMake。包既可以在所谓的经典模式下调用vcpkg手动安装,也可以在清单模式下直接从CMake安装。使用经典方式安装vcpkg包的命令如下:

\begin{tcblisting}{commandshell={}}
vcpkg install [packages]
\end{tcblisting}

当以清单模式运行时,项目的依赖项在vcpkg.json中定义,文件在项目的根目录下。清单模式有一个很大的优势,可以更好地与CMake集成,因此请尽可能使用清单模式。vcpkg清单可能是这样的:

\begin{lstlisting}[style=styleCMake]
{
	"name" : "vcpkg-example",
	"version-semver" : "0.0.1",
	"dependencies" :
	[
	"someLibrary",
	"anotherLibrary",
	] 
}
\end{lstlisting}

为了让CMake找到这些包,vcpkg工具链文件必须传递给CMake,对CMake的使用如下所示:

\begin{tcblisting}{commandshell={}}
cmake -S <source_dir> -D <binary_dir> -DCMAKE_TOOLCHAIN_
  FILE=[vcpkg root]/scripts/buildsystems/vcpkg.cmake
\end{tcblisting}

若以清单模式运行,则vcpkg.json文件中指定的包将自动下载并安装在本地。若以经典模式运行,则必须在运行CMake之前手动安装这些包。当传递vcpkg工具链文件时,可以使用\texttt{find\_package}和\texttt{target\_link\_libraries}使用已安装的包。

微软建议将vcpkg作为子模块安装在存储库中,与CMake根项目处于同一级别,可以安装在任何地方。设置工具链文件可能会在交叉编译时导致问题,因为CMAKE\_TOOLCHAIN\_FILE可能已经指向一个不同的文件,所以第二个工具链文件可以通过VCPKG\_CHAINLOAD\_TOOLCHAIN\_FILE传递:

\begin{tcblisting}{commandshell={}}
cmake -S <source_dir> -D <binary_dir> -DCMAKE_TOOLCHAIN_
  FILE=[vcpkg root]/scripts/buildsystems/vcpkg.cmake -DVCPKG
  _CHAINLOAD_TOOLCHAIN_FILE=/path/to/other/toolchain.cmake
\end{tcblisting}

Conan和vcpkg只是C++和CMake中流行的两个包管理器。当然,还有更多知识点这里没有提及,但需要单独的书来描述它们。特别是当项目变得更加复杂时,强烈建议使用包管理器。

选择哪个包管理器取决于开发项目的环境和个人偏好。Conan比vcpkg有一点优势,因为它在更多的平台上得到支持,可以在Python运行的地方运行。就交叉编译的特性和能力而言,两者不相上下。总之,Conan提供了更高级的配置选项和对包的控制,这是以更复杂的处理为代价的。处理本地依赖关系的另一种方法是通过使用容器、sysroot等创建完全隔离的构建环境,这将在第12章中讨论。现在,假设我们正在使用标准系统安装运行CMake。

处理特定于项目的依赖项时,推荐使用包管理器进行依赖项管理。然而,有时无法选择使用包管理器。这可能是因为公司政策或其他原因。这种情况下,CMake还支持将下载依赖项作为源,并将它们作为外部目标集成到项目中。




