So far, in this book, we have covered how to build and install our own code with CMake. In this chapter, we will take a look at how to use files, libraries, and programs that are not part of a CMake project. The first part of the chapter will be about how to find those things in general, while the latter part will focus on how to manage dependencies to build your CMake project.

One of the biggest advantages of using CMake is that it has built-in dependency management for the discovery of many third-party libraries. In this chapter, we will look at how to integrate libraries that are installed on your system and locally downloaded dependencies. Additionally, you will learn how third-party libraries can be downloaded and used as binaries and, alternatively, how they can be built from source directly out of a CMake project.

We will look at how to write instructions for CMake to reliably find almost any library on your system. Finally, we will take a look at how to use package managers such as Conan and vcpkg with CMake. The practices for dependency management, as covered in this chapter, will help you to create stable and portable builds with CMake. It doesn't matter if you are using precompiled binaries or compiling them in place from scratch, setting up CMake to handle dependencies in a structured and consistent way will reduce the time spent fixing broken builds in the future. Here's the list of main topics that we'll cover in this chapter:

\begin{itemize}
\item 
Finding files, programs, and paths with CMake

\item 
Using third-party libraries in your CMake project

\item 
Using package managers with CMake

\item 
Getting the dependencies as source code
\end{itemize}













