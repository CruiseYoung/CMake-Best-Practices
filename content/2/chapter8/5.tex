构建软件时,常见的任务是在构建前将一些文件复制到特定位置或进行修改。在配置时执行自定义任务一节中,示例会检索Git提交,并将其作为宏定义传递给编译器。一种更好的方法是生成一个包含必要信息的头文件。虽然只是代码片段,并将其写入文件可能导致代码特定于平台,从而很危险。CMake对此的解决方案是使用\texttt{configure\_file}指令,可以将文件从一个位置复制到另一个位置,并在此过程中修改其内容,其签名如下:

\begin{lstlisting}[style=styleCMake]
configure_file(<input> <output>
				NO_SOURCE_PERMISSIONS | USE_SOURCE_PERMISSIONS |
					FILE_PERMISSIONS <permissions>...]
				[COPYONLY] [ESCAPE_QUOTES] [@ONLY]
				[NEWLINE_STYLE [UNIX|DOS|WIN32|LF|CRLF] ])
\end{lstlisting}

\texttt{configure\_file}将<input>文件复制到<output>文件,将创建输出文件的路径,路径可以是相对路径或绝对路径。若使用相对路径,则将从当前源目录搜索输入文件,但输出文件的路径将相对于当前构建目录。若不能写入输出文件,命令失败,配置停止。通常,输出文件与目标文件具有相同的权限,若当前用户与输入文件所属的用户不同,所有权可能会发生变化。若添加了NO\_SOURCE\_PERMISSION,则权限不会转移,输出文件将获得默认权限:rw-r-{}-r-{}-。也可以使用FILE\_PERMISSIONS选项手动指定权限,该选项接受一个三位数的数字作为参数。USE\_SOURCE\_PERMISSION已经是默认选项,这个选项只是为了更明确地进行说明。

除非传递COPYONLY,否则\texttt{configure\_file}也会在复制到输出路径时替换输入文件的部分内容,\texttt{configure\_file}将用同名变量的值替换所有引用为\$\{SOME\_VARIABLE\}或@SOME\_VARIABLE@的变量。若在CMakeLists.txt中定义了一个变量,当使用\texttt{configure\_file}时,相应的值会写入输出文件。若未指定变量,则输出文件将在相应的位置包含一个空字符串。若有一个hello.txt.in文件,包含以下信息:

\begin{lstlisting}[style=styleCMake]
Hello ${GUEST} from @GREETER@
\end{lstlisting}

CMakeLists.txt文件中,\texttt{configure\_file}可用来配置hello.txt.in文件:

\begin{lstlisting}[style=styleCMake]
set(GUEST "World")
set(GREETER "The Universe")
configure_file(hello.txt.in hello.txt)
\end{lstlisting}

生成的hello.txt文件的内容为Hello World from The
Universe。若使用@ONLY选项,则只替换@GREETER@,结果内容将是Hello \$\{GUEST\} from The
Universe。转换CMake文件时,@ONLY很有用,因为CMake文件可能包含不应该被括号括起来的变量。ESCAPE\_QUOTES将用反斜杠转义目标文件中的引号。\texttt{configure\_file}将转换换行符,以便目标文件与当前平台匹配。默认的行为可以通过NEWLINE\_STYLE来改变。UNIX或LF将使用\verb|\|n作为换行符,而DOS、WIN32和CRLF将使用\verb|\|r\verb|\|n。同时设置NEWLINE\_STYLE和COPYONLY选项会导致错误,而设置COPYONLY不会影响换行符的样式。

回到示例,我们将写哈希值写入一个头文件作为输入:

\begin{lstlisting}[style=styleCMake]
#define CMAKE_BEST_PRACTICES_VERSION "@GIT_REVISION@"
The CMakeLists.txt could look something like this:
execute_process(
	COMMAND ${GIT_EXECUTABLE} rev-parse --short HEAD
	OUTPUT_VARIABLE GIT_REVISION
	OUTPUT_STRIP_TRAILING_WHITESPACE
	COMMAND_ERROR_IS_FATAL ANY
	WORKING_DIRECTORY ${CMAKE_CURRENT_SOURCE_DIR})

configure_file(version.h.in
	${CMAKE_CURRENT_SOURCE_DIR}/src/version.h @ONLY)
\end{lstlisting}

版本信息作为宏定义传递,Git哈希信息首先使用\texttt{execute\_process}进行检索。之后,\texttt{configure\_file}复制文件,@GIT\_REVISION@会替换为当前提交的短哈希值。

使用预处理器定义时,\texttt{configure\_file}将替换所有以\#cmakedefine VAR…形式出现的行。使用\#define VAR或/* undef VAR */,这取决于VAR是否包含解释为true或false的值。

有一个名为version.in.h的文件,包含以下两行:

\begin{lstlisting}[style=styleCMake]
#cmakedefine GIT_VERSION_ENABLE
#cmakedefine GIT_VERSION "@GIT_REVISION@"
\end{lstlisting}

附带的CMakeLists.txt文件看起来像这样:

\begin{lstlisting}[style=styleCMake]
option(GIT_VERSION_ENABLE "Define revision in a header file"
	ON)
if(GIT_VERSION_ENABLE)
	execute_process(
		COMMAND ${GIT_EXECUTABLE} rev-parse --short HEAD
		OUTPUT_VARIABLE GIT_REVISION
		WORKING_DIRECTORY ${CMAKE_CURRENT_SOURCE_DIR}
	)
endif()

configure_file(version.h.in
	${CMAKE_CURRENT_SOURCE_DIR}/src/version.h @ONLY)
\end{lstlisting}

配置结束后,若GIT\_REVISION\_ENABLE处于启用状态,结果文件将包含以下内容:

\begin{lstlisting}[style=styleCXX]
#define GIT_VERSION_ENABLE
#define CMAKE_BEST_PRACTICES_VERSION "c030d83"
\end{lstlisting}

若关闭GIT\_REVISION\_ENABLE,生成文件将包含以下内容:

\begin{lstlisting}[style=styleCXX]
/* #undef GIT_VERSION_ENABLE */
/* #undef GIT_REVISION */
\end{lstlisting}

总之,\texttt{configure\_file}对于为构建准备输入非常有用。除了生成源文件外,还经常用于生成CMake文件,然后将这些文件包含在CMakeLists.txt文件中。这样做的优点是允许独立于平台的文件复制和修改,这在跨平台工作时很重要。因为\texttt{configure\_file}和\texttt{execute\_process}通常同时进行,所以要确保执行的命令也是平台相关的。下一节中,将了解如何使用CMake定义与平台无关的命令和脚本。






















