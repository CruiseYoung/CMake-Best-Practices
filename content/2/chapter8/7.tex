本章中,学习了如何通过执行外部任务和程序自定义构建。介绍了如何将自定义构建操作添加为目标,如何将它们添加到现有目标,以及如何在配置步骤中执行。还讨论了命令如何生成文件,以及CMake如何使用\texttt{configure\_file}复制和修改文件。最后,学习了如何使用CMake命令行实用程序以独立于平台的方式执行任务。

自定义CMake构建的能力是一个非常强大的工具,但也使构建更加脆弱,因为在执行自定义任务时,构建的复杂性也会增加。尽管有时不可避免,但依赖于安装的编译器和链接器以外的外部程序,可能会让构建在一个没有安装这些程序或不可用的平台上失败。必须特别小心,以确保自定义任务不会假设系统的情况。最后,执行自定义任务可能会给构建系统带来性能损失。

但若小心使用自定义构建步骤,它们是增加构建内聚性的好方法,因为许多与构建相关的任务都可以在构建定义所在的位置定义。这可以使任务自动化,如创建构建工件的哈希或在公共归档文件中组装所有文档,变得更加容易。

下一章中,将学习如何使构建环境在不同系统之间可移植。了解如何使用预设来定义配置CMake项目的通用方法,以及如何将构建环境包装到容器中,以及如何使用sysroot来定义工具链和库,以便在系统之间移植。