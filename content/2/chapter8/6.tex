
CMake成功的关键是它允许在多种平台上构建相同的软件,同样其也必须以一种不假定使用平台或编译器的方式编写。这很有挑战性,特别是在处理自定义任务时。cmake命令行程序支持-E标志,可以用于执行常见任务,如文件操作和创建哈希值。大多数cmake -E命令用于与文件相关的操作,如创建、复制、重命名和删除文件,以及创建目录。支持文件系统链接的系统,CMake还可以在文件之间创建符号软硬链接。此外,CMake可以使用tar命令创建文件包,并使用cat命令连接文本文件,还可以用于为文件创建各种哈希值。

还有一些操作可以提供关于当前系统的信息。功能操作将打印出CMake的功能,例如:支持哪些生成器,以及CMake当前运行的版本。environment命令将打印已设置的环境变量的列表。

通过不带参数的运行cmake -E,可以获得对命令行选项的完整信息。CMake的在线文档可参见\url{https://cmake.org/cmake/help/v3.21/manual/cmake.1.html#run-a-commandline-tool}。

\begin{tcolorbox}[colback=blue!5!white,colframe=blue!75!black,title=平台无关的文件操作]
当文件操作必须由自定义任务执行时,可以使用cmake -E。
\end{tcolorbox}

大多数情况下,可以用cmake -E,有时需要进行更复杂的操作。为此,CMake可以以脚本模式运行。

\subsubsubsection{8.6.1\hspace{0.2cm}CMake文件作为脚本执行}

创建跨平台脚本时,CMake的脚本模式是非常强大的特性,因为允许创建完全与平台无关的脚本。通过调用cmake -P <script>.cmake,则执行指定的CMake文件。脚本文件可能不包含定义构建目标的命令。参数可以作为带有-D标志的变量传递,但这必须在-P选项之前完成。或者,参数可能只放在脚本名称之后,这样就可以使用CMAKE\_ARGV[n]检索它们,并且参数的数量存储在CMAKE\_ARGC变量中。下面的脚本生成一个文件的哈希码,并将其存储在另一个文件中:

\begin{lstlisting}[style=styleCMake]
cmake_minimum_required(VERSION 3.21)
if(CMAKE_ARGC LESS 5)
	message(FATAL_ERROR "Usage: cmake -P CreateSha256.cmake
		file_to_hash target_file")
endif()

set(FILE_TO_HASH ${CMAKE_ARGV3})
set(TARGET_FILE ${CMAKE_ARGV4})

# Read the source file and generate the hash for it
file(SHA256 "${FILE_TO_HASH}" GENERATED_HASH)

# write the hash to a new file
file(WRITE "${TARGET_FILE}" "${GENERATED_HASH}")
\end{lstlisting}

这个脚本的使用方式为cmake -P CreateSha256.cmake <input\_file> <output\_file>。注意,前三个参数是cmake、-P和脚本的名称(CreateSha256.cmake)。尽管不是严格要求的,脚本文件应该总是在开头包含\texttt{cmake\_minimum\_required}。另一种不带位置参数定义脚本的方法如下所示:

\begin{lstlisting}[style=styleCMake]
cmake_minimum_required(VERSION 3.21)
if(NOT FILE_TO_HASH OR NOT TARGET_FILE)
	message(FATAL_ERROR "Usage: cmake –DFILE_TO_HASH=<intput_file> 
		-DTARGET_FILE=<target file> -P CreateSha256.cmake")
endif()

# Read the source file and generate the hash for it
file(SHA256 "${FILE_TO_HASH}" GENERATED_HASH)

# write the hash to a new file
file(WRITE "${TARGET_FILE}" "${GENERATED_HASH}")
\end{lstlisting}

必须使用显式传递的变量调用脚本,如下所示:

\begin{tcblisting}{commandshell={}}
cmake –DFILE_TO_HASH=<input>
      -DTARGET_FILE=<target> -P CreateSha256.cmake
\end{tcblisting}

这两种方法也可以结合使用。一种常见的模式是期望所有简单的强制参数作为位置参数,可选或更复杂的参数作为已定义变量。将脚本模式与\texttt{add\_custom\_command}、\texttt{add\_custom\_target}或\texttt{execute\_process}相结合,可以创建独立于平台的构建指令。从前面生成哈希值的示例如下所示:

\begin{lstlisting}[style=styleCMake]
add_custom_target(Create_hash_target ALL
COMMAND cmake -P ${CMAKE_CURRENT_SOURCE_DIR}/cmake/CreateSha256.cmake
	$<TARGET_FILE:SomeTarget>
	${CMAKE_CURRENT_BINARY_DIR}/hash_example.sha256
)

add_custom_command(TARGET SomeTarget
POST_BUILD
COMMAND cmake -P ${CMAKE_CURRENT_SOURCE_DIR}/cmake/CreateSha256.cmake
	$<TARGET_FILE:SomeTarget>
	${CMAKE_CURRENT_BINARY_DIR}/hash_example.sha256
)
\end{lstlisting}

将CMake的脚本模式,与在项目的配置或构建阶段执行定制命令的方法相结合,可以在定义构建过程时提供很大的自由度,甚至可以针对不同的平台。然而,在构建过程中添加太多的逻辑可能会使其难于维护。当需要编写脚本或向CMakeLists.txt文件添加自定义命令时,最好是考虑一下,这一步是否属于构建过程,还是在用户设置开发环境时让他们设置。





























