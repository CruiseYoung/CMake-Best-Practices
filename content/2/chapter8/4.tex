
要在配置时执行自定义任务,可以使用\texttt{execute\_process}:生成在构建前需要的信息,或者需要更新文件以重新运行CMake。另一个情况是在配置步骤中生成CMakeLists.txt文件或其他输入文件,也可以通过\texttt{configure\_file}实现。

\texttt{execute\_process}的工作原理与\texttt{add\_custom\_target}和\texttt{add\_custom\_command}非常相似。然而,区别是\texttt{execute\_process}可以在变量或文件中捕获到stdout和stderr的输出。\texttt{execute\_process}的签名如下:

\begin{lstlisting}[style=styleCMake]
execute_process(COMMAND <cmd1> [<arguments>]
				[COMMAND <cmd2> [<arguments>]]...
				[WORKING_DIRECTORY <directory>]
				[TIMEOUT <seconds>]
				[RESULT_VARIABLE <variable>]
				[RESULTS_VARIABLE <variable>]
				[OUTPUT_VARIABLE <variable>]
				[ERROR_VARIABLE <variable>]
				[INPUT_FILE <file>]
				[OUTPUT_FILE <file>]
				[ERROR_FILE <file>]
				[OUTPUT_QUIET]
				[ERROR_QUIET]
				[COMMAND_ECHO <where>]
				[OUTPUT_STRIP_TRAILING_WHITESPACE]
				[ERROR_STRIP_TRAILING_WHITESPACE]
				[ENCODING <name>]
				[ECHO_OUTPUT_VARIABLE]
				[ECHO_ERROR_VARIABLE]
				[COMMAND_ERROR_IS_FATAL <ANY|LAST>])
\end{lstlisting}

\texttt{execute\_process}接受在WORKING\_DIRECTORY中执行的COMMAND列表。最后一个要执行的命令的返回代码可以存储在RESULT\_VARIABLE定义的变量中,或可以将使用分号分隔的变量列表传递给RESULTS\_VARIABLE。若使用的是列表,则命令将按与已定义的变量相同的顺序存储命令的返回代码。若定义的变量比命令少,那么多余的返回代码都将忽略。若定义了TIMEOUT,有子进程都无法结束时,结果变量将包含超时。自CMake版本3.19以来,COMMAND\_ERROR\_IS\_FATAL选项可用,进程失败(或是最后一个)时中止执行。下面的例子中,若命令返回非零值,CMake的配置步骤将失败并出现错误:

\begin{lstlisting}[style=styleCMake]
execute_process(
	COMMAND SomeExecutable
	COMMAND AnotherExecutable
	COMMAND_ERROR_IS_FATAL_ANY
)
\end{lstlisting}

stdout或stderr的输出可以使用OUTPUT\_VARIABLE或ERROR\_VARIABLE捕获。一种替代方案是,通过使用OUTPUT\_FILE或ERROR\_FILE重定向到文件,也可以通过传递OUTPUT\_QUIET或ERROR\_QUIET忽略。在变量和文件中不可能捕获输出,这将导致两者中的一个为空。哪些保留,哪些丢弃取决于平台。若不重定向,OUTPUT\_*选项可以将指定输出发送给CMake进程。

若使用变量捕获输出,但仍然需要显示,可以添加ECHO\_<STREAM>\_VARIABLE。CMake还可以通过向COMMAND\_ECHO选项传递STDOUT、STDERR或NONE来输出命令。但若输出在文件中捕获,这将没有效果。若为stdout和stderr指定了相同的变量或文件,则将合并结果,并且可以通过INPUT\_FILE选项传递一个文件来控制第一个命令的输入流。

变量的输出可以使用<STREAM>\_STRIP\_TRAILING\_WHITESPACE选项的方式控制,该选项将消除输出末尾的空白。当将输出重定向到文件时,这个选项无效。Windows上,ENCODING选项可用于控制输出,并接受以下值:

\begin{itemize}
\item 
NONE: 不执行重新编码。这将保持CMake的内部编码,即UTF-8。

\item 
AUTO: 使用当前控制台的编码。若不可用,将使用ANSI。

\item 
ANSI: 使用ANSI进行编码。

\item 
OEM: 使用平台定义的编码。

\item 
UTF8或UTF-8: 强制使用UTF-8编码。
\end{itemize}

使用\texttt{execute\_process}的原因是收集构建所需的信息,然后将其传递给项目。假设希望通过传递宏定义来将Git修订编译为可执行文件。这种方法的缺点是,对于要执行的定制任务,必须调用CMake,而不仅仅是构建系统。因此,使用带OUTPUT参数的使用\texttt{add\_custom\_command}可能是更现实的解决方案,但出于演示目的,这个示例应该足够了。以下是在配置时读取Git哈希值,并将其作为编译定义传递给目标的示例:

\begin{lstlisting}[style=styleCMake]
find_package(Git REQUIRED)
execute_process(COMMAND ${GIT_EXECUTABLE} "rev-parse" "--short"
	"HEAD"
	OUTPUT_VARIABLE GIT_REVISION
	OUTPUT_STRIP_TRAILING_WHITESPACE
	COMMAND_ERROR_IS_FATAL ANY
	WORKING_DIRECTORY ${CMAKE_CURRENT_SOURCE_DIR})

add_executable(SomeExe src/main.cpp)
target_compile_definitions(SomeExe PRIVATE VERSION=\"${GIT_REVISION}\")
\end{lstlisting}

传递给\texttt{execute\_process}的git命令将在当前执行CMakeLists.txt文件的目录中执行。生成的哈希存储在GIT\_REVISION变量中,若命令因任何原因失败,配置过程将停止并出现错误。

通过宏定义将信息从\texttt{execute\_process}传递到编译器,并不是最优的方式。更好的解决方案是,生成一个包含此信息的头文件。CMake还有\texttt{configure\_file}指令可以用于此目的,我们将在下一节中看到。

















