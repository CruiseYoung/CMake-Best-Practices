测试是以软件质量为傲的软件工程师的主要方式。目前已有很多单元测试框架,特别是对于C++, CMake包含了可与多数流行框架一起使用的模块。

总的来说,所有单元测试框架都会做以下工作:

\begin{itemize}
\item 
制定和分组测试用例。

\item 
包含某种形式的断言,以检查各种测试条件。

\item 
发现并运行测试用例,可以是全部的,也可以是其中的一个选择。

\item 
生成各种格式的测试结果,例如纯文本、JSON、XML,可能还有更多。
\end{itemize}

通过CTest,CMake可以以内置的方法来执行测试。设置\texttt{enable\_testing()},并使用\texttt{add\_test()}添加了测试的CMake项目都支持运行测试。\texttt{enable\_testing()}将在当前目录和其子目录中启用,并添加测试。因此在使用\texttt{add\_subdirectory}前,通常将其设置在顶层的\texttt{CMakeLists.txt}中。若使用\texttt{include(CTest)},CMake的CTest模块会自动设置enable\_testing,除非BUILD\_TESTING为OFF。

根据BUILD\_TESTING选项禁用构建和运行测试也是一种很好的实践。这里的常用模式是将项目中与测试相关的所有部分放在其子文件夹中,并且只在BUILD\_TESTING设置为ON时包含子文件夹。

CTest模块通常应该只包含在项目的顶层CMakeLists.txt中。自从CMake版本3.21以来,PROJECT\_IS\_TOP\_LEVEL可以用来测试当前的CMakeLists.txt是否为顶层文件。对于项目的顶层目录和使用\texttt{ExternalProject}添加的项目顶层目录,此变量为True。对于使用\texttt{add\_subdirectory}或\texttt{FetchContent}添加的目录,该值为False。因此,CTest应该包括如下内容:

\begin{lstlisting}[style=styleCMake]
project(CMakeBestPractice)
...
if(PROJECT_IS_TOP_LEVEL)
	include(CTest)
endif()
\end{lstlisting}

单元测试本质上是在内部运行一系列断言的小程序,若断言失败,将返回一个非零返回值。有许多框架和库可以帮助组织测试和编写断言,但从外部来看,检查断言和返回相应的值是核心功能。

可以使用\texttt{add\_test}将测试添加到CMakeLists.txt中:

\begin{lstlisting}[style=styleCMake]
add_test(NAME <name> COMMAND <command> [<arg>...]
	[CONFIGURATIONS <config>...]
	[WORKING_DIRECTORY <dir>]
	[COMMAND_EXPAND_LISTS])
\end{lstlisting}

COMMAND可以是项目中定义的可执行目标,也可以是可执行文件的完整路径,测试所需的参数也包括在内。因为CMake会自动替换到可执行文件的路径,所以使用目标名称为首选。CONFIGURATION选项用于判断对哪个构建配置有效。对于大多数测试用例来说,这并不重要,但是对于微基准测试来说,就非常有用,WORKING\_DIRECTORY应该是一个绝对路径。默认情况下,测试在CMAKE\_CURRENT\_BINARY\_DIR中执行。COMMAND\_EXPAND\_LISTS确保传递给作为COMMAND选项的列表都会展开。

包含测试的小项目:

\begin{lstlisting}[style=styleCMake]
cmake_minimum_required(VERSION 3.21)
project("simple_test" VERSION 1.0)

enable_testing()

add_executable(simple_test)
target_sources(simple_test PRIVATE src/main.cpp)

add_test(NAME example_test COMMAND simple_test)
\end{lstlisting}

本例中,一个名为simple\_test的可执行目标用作example\_test的测试。

CTest将使用有关测试的信息并执行它们。通过独立运行ctest或作为CMake构建步骤的一部分,来执行测试。以下两个命令都可执行测试:

\begin{tcblisting}{commandshell={}}
ctest --test-dir <build_dir>
cmake --build <build_dir> --target test
\end{tcblisting}

调用CTest作为构建目标的好处是,CMake会首先检查所有需要的目标是否构建了,并且在最新的版本上CTest也可以这样做:

\begin{tcblisting}{commandshell={}}
ctest --build-and-test <source_dir> <build_dir>
\end{tcblisting}

ctest的输出看起来可能像这样:

\begin{tcblisting}{commandshell={}}
Test project /workspaces/CMake-Best-Practices/build
    Start 1: example_test
1/3 Test #1: example_test .....................***Failed
0.00 sec
    Start 2: pass_fail_test
2/3 Test #2: pass_fail_test ................... Passed
\end{tcblisting}
\begin{tcblisting}{commandshell={}}
0.00 sec
    Start 3: timeout_test
3/3 Test #3: timeout_test ..................... Passed
0.50 sec

67% tests passed, 1 tests failed out of 3

Total Test time (real) = 0.51 sec

The following tests FAILED:
         1 - example_test (Failed)
Errors while running CTest
Output from these tests are in:
     /workspaces/CMake-Best-Practices/build/Testing/Temporary/LastTest.log
Use "--rerun-failed --output-on-failure" to re-run the failed
     cases verbosely.
\end{tcblisting}

通常,测试静默所有输出到标准输出。通过-V或-{}-verbose命令行参数,可将测试输出到屏幕。但通常只对失败测试的输出感兴趣,使用-{}-output-on-failure通常是更好的选择,只有失败的测试才会产生输出。对于非常详细的测试,可以使用-{}-test-output-size-passed <size>和-{}-test-output-size-failed <size>选项来限制输出的大小,其中大小是字节(byte)数。

构建树中有一个或多个add\_test将会在CMAKE\_CURRENT\_BINARY\_DIR中为CTest生成一个输入文件。CTest的输入文件不一定位于项目的顶层,而是在定义的地方。要列出所有测试但不执行,可以使用CTest的-N选项。

CTest的特性会在运行之间缓存测试状态,可以只运行上次运行失败的测试。为此,运行ctest -{}-rerun-failed将只运行上次运行中失败的测试。

有时,若要修复单个失败的测试,则不希望执行完整的测试集。-E和-R命令行选项接受与测试名称匹配的正则表达式(正则表达式)。-E选项排除匹配模式的测试,-R选项选择要包含的测试。这些选项可以组合使用。以下命令将运行所有以FeatureX开头的测试,但不包括名为FeatureX\_Test\_1的测试:

\begin{tcblisting}{commandshell={}}
ctest -R ^FeatureX -E FeatureX_Test_1
\end{tcblisting}

有选择地执行测试的另一种方法是使用LABELS属性进行标记,然后使用CTest的-L选项选择要运行的标签。一个测试可以有多个用分号分隔的标签:

\begin{lstlisting}[style=styleCMake]
add_test(NAME labeled_test_1 COMMAND someTest)
set_tests_properties(labeled_test PROPERTIES LABELS "example")

add_test(NAME labeled_test_2 COMMAND anotherTest)
set_tests_properties(labeled_test_2 PROPERTIES LABELS "will_fail" )

add_test(NAME labeled_test_3 COMMAND YetAnotherText)
set_tests_properties(labeled_test_3
	PROPERTIES LABELS "example;will_fail")
\end{lstlisting}

-L命令行选项接受一个正则表达式来过滤标签:

\begin{tcblisting}{commandshell={}}
ctest -L example
\end{tcblisting}

这将只执行labeled\_test\_1和labeled\_test\_3,因为它们都分配了example标签。

通过制定相应的正则表达式,可以组合多个标签:

\begin{tcblisting}{commandshell={}}
ctest -L "example|will_fail"
\end{tcblisting}

这将执行示例中的所有测试,但不会执行没有分配标签的测试。

使用标签将有助于标记设计为失败或类似的测试,或标记仅与某些执行上下文中相关的测试。

对于基于正则表达式或标签的测试选择,最后一种选择是使用-I选项,它接受分配的测试编号。-I选项的参数有点复杂:

\begin{tcblisting}{commandshell={}}
ctest -I [Start,End,Stride,test#,test#,...|Test file]
\end{tcblisting}

通过Start、End和Stride,可以指定要执行的测试的范围。这三个数字是与显式测试数字test\#相结合的范围,或传递包含参数的文件。

下面的调用将执行从1到10的所有奇数测试:

\begin{tcblisting}{commandshell={}}
ctest -I 1,10,2
\end{tcblisting}

因此,将执行测试1、3、5、7和9。以下命令将只执行测试和8:

\begin{tcblisting}{commandshell={}}
ctest -I ,0,,6,7
\end{tcblisting}

这个调用中,End设置为0,因此没有执行。要合并范围和显式测试数,以下命令将执行从1到10的所有奇数测试,并额外执行测试6和8:

\begin{tcblisting}{commandshell={}}
ctest -I 1,10,2,6,8
\end{tcblisting}

-I选项处理繁琐,加上添加新测试可能会重新分配数字,这是在实践中很少使用它的两个原因。通常,根据标签或测试名称进行过滤是首选方式。

编写测试时的另一个常见问题是它们不够独立。因此,test 2可能会意外地依赖于test 1的前一次执行。为了避免这种依赖,CTest可使用-{}-schedule-random参数随机化测试执行顺序。这将确保测试以任意顺序执行。

\subsubsubsection{7.2.1\hspace{0.2cm}自动发现测试}

\texttt{add\_test}定义测试是向CTest添加测试的一种方法,缺点是这会将整个可执行文件注册为单个测试。通常,单个可执行文件将包含多个单元测试,不仅仅是一个。当可执行文件中的一个测试失败时,可能很难确定到底哪个测试失败了。

以下测试代码的C++文件,假设Fibonacci函数包含一个错误,那么Fibonacci(0)将不会像它应该返回的那样返回1:

\begin{lstlisting}[style=styleCXX]
TEST_CASE("Fibonacci(0) returns 1"){ REQUIRE(Fibonacci(0) == 1);}
TEST_CASE("Fibonacci(1) returns 1"){ REQUIRE(Fibonacci(1) == 1); }
TEST_CASE("Fibonacci(2) returns 2"){ REQUIRE(Fibonacci(2) == 2); }
TEST_CASE("Fibonacci(5) returns 8"){ REQUIRE(Fibonacci(5) == 8); }
\end{lstlisting}

若所有这些测试都编译到同一个名为Fibonacci的可执行文件中,那么使用\texttt{add\_test}添加只会指示可执行文件失败,而不会指示前面代码块中看到的哪个场景失败。

测试结果看起来像这样:

\begin{tcblisting}{commandshell={}}
Test project /workspaces/CMake-Best-Practices/build
    Start 5: Fibonacci
1/1 Test #5: Fibonacci ........................***Failed
0.00 sec
0% tests passed, 1 tests failed out of 1
Total Test time (real) = 0.01 sec
The following tests FAILED:
         5 - Fibonacci (Failed)
\end{tcblisting}

这对找出哪个测试用例失败没什么帮助,但在Catch2和GoogleTest中,有一种方法可以将内部测试公开给CTest,从而将它们作为常规测试执行。对于GoogleTest,这样做的模块是由CMake本身提供的;Catch2在自己的CMake集成中提供了这个功能。使用Catch2发现测试由\texttt{catch\_discover\_tests}完成,而对于GoogleTest,则使用\texttt{gtest\_discover\_tests}。以下示例在Catch2框架中编写的测试,并添加到CTest:

\begin{lstlisting}[style=styleCMake]
find_package(Catch2)
include(Catch)

add_executable(Fibonacci)
catch_discover_tests(Fibonacci)
\end{lstlisting}

为了使函数可用,必须包含Catch模块。对于GoogleTest,工作原理相似:

\begin{lstlisting}[style=styleCMake]
include(GoogleTest)

add_executable(Fibonacci)
gtest_discover_tests(Fibonacci)
\end{lstlisting}

使用发现功能时,测试可执行文件中定义的每个测试用例都将视为CTest自己的测试,CTest的结果可能如下所示:

\begin{tcblisting}{commandshell={}}
    Start 5: Fibonacci(0) returns 1
1/4 Test #5: Fibonacci(0) returns 1 .........***Failed 0.00 sec
    Start 6: Fibonacci(1) returns 1
2/4 Test #6: Fibonacci(1) returns 1 ......... Passed 0.00 sec
    Start 7: Fibonacci(2) returns 2
3/4 Test #7: Fibonacci(2) returns 2 ......... Passed 0.00 sec
    Start 8: Fibonacci(5) returns 8
4/4 Test #8: Fibonacci(5) returns 8 ......... Passed 0.00 sec

75% tests passed, 1 tests failed out of 4
Total Test time (real) = 0.02 sec
The following tests FAILED:
         5 - Fibonacci(0) returns 1 (Failed)
\end{tcblisting}

现在,可以看到哪些定义的测试用例失败了,Fibonacci(0)返回1个测试用例没有像预期的那样运行。当使用带有集成测试功能的编辑器或IDE时,这很方便。这两个发现函数都通过运行指定的可执行文件来工作,该可执行文件有一个选项,即只打印测试名称并在CTest内部注册,因此每个构建步骤都会增加少量开销。更细粒度地发现测试还有一个好处,即CMake可以更好地并行执行测试。

\texttt{gtest\_discover\_tests}和\texttt{catch\_discover\_tests}也有丰富的选项,比如为测试名称添加前缀或后缀,或者为生成的测试添加一组属性。函数的完整文档可以在这里找到:

\begin{itemize}
\item 
Catch2: \url{https://github.com/catchorg/Catch2/blob/devel/docs/cmake-integration.md}
	
\item 
GoogleTest: \url{https://cmake.org/cmake/help/v3.21/module/GoogleTest.html}
\end{itemize}

Catch2和GoogleTest只是众多测试框架中的两个;可能有更多的测试套件自带这种功能,而这些功能可能是作者不知道的。现在,让我们从寻找测试开始,进一步了解如何控制测试的行为。

\subsubsubsection{7.2.2\hspace{0.2cm}确定测试成功或失败的高级方法}
 
通常,CTest根据命令的返回值确定测试是失败还是通过。0表示所有测试都成功,否则将解释为失败。

有时,返回值不足以确定测试通过或失败。若需要检查某个字符串的程序输出,可以使用FAIL\_REGULAR\_EXPRESSION和PASS\_REGULAR\_EXPRESSION测试属性:

\begin{lstlisting}[style=styleCMake]
set_tests_properties(some_test PROPERTIES
					FAIL_REGULAR_EXPRESSION "[W|w]arning|[E|e]rror"
					PASS_REGULAR_EXPRESSION "[S|s]uccess")
\end{lstlisting}

若输出包含“Warning”或“Error”,这些属性将导致some\_test测试失败。若有"Success"字符串,则认为测试通过。如果设置了PASS\_REGULAR\_EXPRESSION,则只有该字符串存在,测试才会认为通过。这两种情况下,将忽略返回值。若需要忽略测试的某个返回值,可以使用SKIP\_RETURN\_CODE选项。

有时,会需要测试失败,设置WILL\_FAIL为true将导致测试结果颠倒:

\begin{lstlisting}[style=styleCMake]
add_test(NAME SomeFailingTerst COMMAND SomeFailingTest)
set_tests_properties(SomeFailingTest PROPERTIES WILL_FAIL True)
\end{lstlisting}

这通常比禁用测试要好,因为测试在每次测试运行时仍然会执行,若测试意外地再次开始通过,开发人员会知道。测试失败的一种特殊情况,是当测试无法返回或需要花费太多时间才能完成。对于这种情况,CTest提供了添加测试超时的方法,甚至在失败的情况下重新测试。

\subsubsubsection{7.2.3\hspace{0.2cm}处理超时和重复的测试}

有时,不仅对测试的成功或失败感兴趣,还对完成测试需要多长时间感兴趣。TIMEOUT测试属性需要数秒来确定测试的最大运行时。若测试超过时间,则终止测试并认为测试失败。以下命令将限制测试的测试执行时间为10秒:

\begin{lstlisting}[style=styleCMake]
set_tests_properties(timeout_test PROPERTIES TIMEOUT 10)
\end{lstlisting}

对于有陷入无限循环或由于某种原因永远挂起的风险的测试,TIMEOUT属性就很好用。

或者,CTest接受-{}-timeout参数来设置全局超时时限,该时限应用于没有指定TIMEOUT属性的所有测试。对于那些定义了TIMEOUT的测试,CMakeLists.txt中定义的超时,优先于通过命令行传递的时限。

为了避免长时间的测试执行,CTest命令行接受-{}-stop-time参数,将当天的实时作为完整测试集的时间限制。下面的命令将为每个测试设置一个默认的30秒超时,并且测试必须在23:59之前完成:

\begin{tcblisting}{commandshell={}}
ctest --timeout 30 --stop-time 23:59
\end{tcblisting}

有时,由于我们无法控制的因素,测试可能会偶尔出现超时。通常是需要某种形式的网络通信或具有某种带宽限制的资源的测试,让测试运行的唯一方法是再次尝试。为此,-{}-repeat aftertimeout:n命令行参数可以传递给CTest,其中n是一个数字。

-{}-repeat参数有三个选项:

\begin{itemize}
\item 
after-timeout: 若发生超时,这将重试测试多次。当超时后重复测试,-{}-timeout选项应该传递给CTest。

\item 
until-pass: 这将重新运行测试,直到测试通过或达到重试次数为止,将此设置为CI环境中的规则并不是一个好主意。

\item 
until-fail: 测试将重新运行多次,直到它们失败。若测试偶尔失败,则经常使用此方法,以了解这种情况发生的频率。-{}-repeatuntil-fail参数的工作原理完全类似于-{}-repeat:until-fail:n。
\end{itemize}

如前所述,测试失败的原因可能是测试所依赖的资源不可用。外部资源不可用的情况是它们被来自测试的请求淹没。访问外部资源时超时的另一个常见原因是,当测试运行时该资源不可用。下一节中,将了解如何编写用于确保在测试运行之前启动资源的测试固件。

\subsubsubsection{7.2.4\hspace{0.2cm}编写测试固件}

通常,测试应该彼此独立。有时,测试可能依赖于不受测试本身控制的前提条件。例如,为了测试客户端,测试可能需要运行一个服务器。这些依赖关系可以通过使用FIXTURE\_SETUP、FIXTURE\_CLEANUP和FIXTURE\_REQUIRED测试属性将它们定义为测试固件。所有三个属性都接受一个字符串列表来标识固件,测试可以通过定义FIXTURE\_REQUIRED属性来指定相应的固件。这将确保固件测试在执行前成功完成。类似地,测试可以在FIXTURE\_CLEANUP中进行声明,以表明必须在该固件测试完成后方可运行。无论测试是成功还是失败,清理部分中定义的固件总是会运行:

\begin{lstlisting}[style=styleCMake]
add_test(NAME start_server COMMAND echo_server --start)
set_tests_properties(start_server PROPERTIES FIXTURES_SETUP server)
add_test(NAME stop_server COMMAND echo_server --stop)
set_tests_properties(stop_server PROPERTIES FIXTURES_CLEANUP server)

add_test(NAME client_test COMMAND echo_client)
set_tests_properties(client_test PROPERTIES FIXTURES_REQUIRED server)
\end{lstlisting}

本例中,名为echo\_server的程序当作固件,以便另一个echo\_client的程序使用它。带有-{}-start和-{}-stop参数的echo\_server的执行名称带有start\_server和stop\_server的测试。start\_server测试标记为带有名称server的固件设置。stop\_server测试也进行了类似的设置,但标记为固件清理例程。最后,将设置名为client\_test的实际测试,并将它作为服务器固件必需通过的先决条件。

现在,若使用CTest运行client\_test测试,则会自动调用固件。固件测试在CTest的输出中显示为常规测试:

\begin{tcblisting}{commandshell={}}
ctest -R client
Test project CMake-Best-Practices:
    Start 9: start_server
1/3 Test #9: start_server .............. Passed 0.00 sec
    Start 11: client_test
\end{tcblisting}
\begin{tcblisting}{commandshell={}}
2/3 Test #11: client_test................ Passed 0.00 sec
    Start 10: stop_server
3/3 Test #10: stop_server ............... Passed 0.00 sec
\end{tcblisting}

CTest用只匹配客户端测试的正则表达式筛选器,但CTest还是会启动固件。为了在并行执行测试时,不使测试固件压力过重,可以将它们定义为资源。

\subsubsubsection{7.2.5\hspace{0.2cm}管理测试资源——并行运行测试}

若项目有许多测试,那并行执行它们将加快测试的速度。默认情况下,CTest按顺序运行测试,通过将-j选项传递给CTest调用,这些测试可以并行运行,或并行线程数量可以在CTEST\_PARALLEL\_LEVEL环境变量中定义。通常,CTest假定每个测试将在单个CPU上运行。若测试需要多个处理器才能成功运行,可以设置测试的PROCESSORS属性来定义所需的处理器数量:

\begin{lstlisting}[style=styleCMake]
add_test(NAME concurrency_test COMMAND concurrency_tests)
set_tests_properties(concurrency_test PROPERTIES PROCESSORS 2)
\end{lstlisting}

concurrency\_test测试需要两个CPU才能运行。当使用-j 8并行运行测试时,concurrency\_test将占用八个可用的并行执行“槽”中的两个。本例中,若PROCESSORS属性设置为8,则没有其他测试可以与concurrency\_test并行运行。当为PROCESSORS设置的值高于系统上可用的并行槽数或CPU数时,就会在有CPU空闲的时候,立即运行测试。有时,有些测试不仅需要特定数量的处理器,而且需要在不运行其他测试的情况下单独运行。为了实现这一点,RUN\_SERIAL属性可以在测试中设置为true。这会对整体测试性能产生严重影响,因此要谨慎使用。更细粒度的控制方法是使用RESOURCE\_LOCK属性,该属性包含一个字符串列表。字符串没有特殊含义,除非CTest阻止两个测试并列运行(若列出的字符串相同)。通过这种方式,可以在不停止整个测试执行的情况下实现部分序列化,这也是指定测试是否需要特定的唯一资源的好方法,例如对某个文件、数据库或类似的东西进行测试时。考虑下面的例子:

\begin{lstlisting}[style=styleCMake]
set_tests_properties(database_test_1 database_test_2
	database_test_3 PROPERTIES RESOURCE_LOCK database)
set_tests_properties(some_other_test PROPERTIES RESOURCE_LOCK
	fileX)
set_tests_properties(yet_another_test PROPERTIES RESOURCE_LOCK
	"database;fileX ")
\end{lstlisting}

本例中,database\_test\_1、database\_test\_2和database\_test\_3测试将禁止并行运行。some\_other\_test测试将不受数据库测试的影响,但yet\_another\_test将不会与数据库测试和some\_other\_test一起运行。

\begin{tcolorbox}[colback=blue!5!white,colframe=blue!75!black,title=固件为源]
虽然技术上不需要,但若RESOURCE\_LOCK与FIXTURE\_SETUP、FIXTURE\_CLEANUP和FIXTURE\_REQUIRED一起使用,那么为相同的资源使用相同的标识符也是一种很好的方式。
\end{tcolorbox}

当测试需要独占访问某些资源时,使用RESOURCE\_LOCK管理测试的并行性非常方便。大多数情况下,管理并行性就完全足够了。从CMake 3.16开始,可以通过RESOURCE\_GROUPS属性在更细粒度的级别上进行控制。资源组不仅允许指定使用什么资源,还允许指定使用多少资源。常见的场景是定义特定贪婪操作可能需要的内存量,或者避免超过特定服务的连接限制。开发使用GPU进行通用计算的项目时,资源组通常会发挥作用,以定义每个测试需要多少个GPU插槽。

与简单的资源锁相比,资源组的复杂性高了很多。要使用它们,CTest必须做以下事情:

\begin{itemize}
\item 
了解测试需要运行哪些资源,是通过设置项目中的测试属性来定义的。

\item 
了解系统有哪些可用资源,是在运行测试时从项目外部完成的。

\item 
传递关于使用哪些资源进行测试的信息,是通过使用环境变量来实现的。
\end{itemize}

与资源锁一样,资源组是用于标识资源的字符串。绑定到标签的实际资源的定义留给用户。资源组定义为名称:值对,若有多个组,则以逗号分隔。测试可以用RESOURCE\_GROUPS属性定义使用什么资源:

\begin{lstlisting}[style=styleCMake]
set_property(TEST SomeTest PROPERTY RESOURCE_GROUPS
	cpus:2,mem_mb:500
	servers:1,clients:1
	servers:1,clients:2
	4,servers:1,clients:1
)
\end{lstlisting}

前面的示例中,SomeTest表示它使用两个CPU和500 MB内存。总共使用了一个客户机-服务器对的6个实例,每对都分配了一些服务器和客户机。第一对包含一个服务器实例和一个客户端实例,第二对需要一个服务器和两个客户端实例。

最后一行,4,servers:1,clients:1,使用同一对的四个实例,由一个服务器资源和一个客户机资源组成。除了所需的CPU和内存外,除非总共有6个服务器和7个客户机可用,否则此测试将不会运行。

可用的系统资源在传递给CTest的JSON文件中指定,可以通过CTest -{}-resource-spec-file命令行参数指定,也可以通过调用CMake时设置CTEST\_RESOURCE\_SPEC\_FILE变量指定。设置变量应该通过使用cmake -D来完成,而不是在CMakeLists.txt中直接设置,指定系统资源应该从项目外部完成。

上面示例的资源指定文件如下所示:

\begin{lstlisting}[style=styleCMake]
{
	"version": {
		"major": 1,
		"minor": 0
	},
	"local": [
	{
		"mem_mb": [
			{
				"id": "memory_pool_0",
				"slots": 4096
			}
		],
		"cpus" :
		[
			{
				"id": "cpu_0",
				"slots": 8
			}
		],	
		"servers": [
			{
				"id": "0",
				"slots": 4
			},
			{
				"id": "1",
				"slots": 4
			}
		],
		"clients": [
		{
			"id": "0",
			"slots": 8
		},
		{
			"id": "1",
			"slots": 8
		}
		]
	}
	]
}
\end{lstlisting}

该文件指定一个系统,共有4,096 MB内存、8个CPU、2x4服务器实例和2x8客户端实例,共包含8个服务器和16个客户端。若测试的资源请求不能满足可用的系统资源,将无法运行测试,会出现如下错误:

\begin{tcblisting}{commandshell={}}
ctest –j $(nproc) --resource-spec-file ../resources.json

Test project /workspaces/CMake-Best-Practices/chapter07
  /resource_group_example/build
     Start 2: resource_test_2
                  Start 3: resource_test_3
Insufficient resources for test resource_test_3:

    Test requested resources of type 'mem_mb' in the following
        amounts:
        8096 slots
    but only the following units were available:
        'memory_pool_0': 4096 slots

Resource spec file:

  ../resources.json
\end{tcblisting}

当前示例将能够在此环境下运行,因为它总共需要6个服务器和7个客户机。CTest没有办法确保指定的资源是否实际可用,这是用户或CI系统需要确定的事情。例如,一个资源文件可能指定有八个可用的CPU,而硬件实际上只有四个核。

关于已分配资源组的信息通过环境变量传递给测试,CTEST\_RESOURCE\_GROUP\_COUNT环境变量指定分配给测试的资源组的总数。若没有设置,则在没有环境文件的情况下使用CTest。测试应该检查这一点并相应地采取行动,若一个测试在没有资源的情况下不能运行,要么失败,要么通过返回各自的返回代码或在SKIP\_RETURN\_CODE或SKIP\_REGULAR\_EXPRESSION属性中定义的字符串来指示它没有运行。分配给测试的资源组通过成对的环境变量指定:

\begin{itemize}
\item 
CTEST\_RESOURCE\_GROUP\_<ID>将包含资源组的类型。前面的示例中,为"mem\_mb"、"cpus"、"clients"或"servers"。

\item 
CTEST\_RESOURCE\_GROUP\_<ID>\_<TYPE>,将包含用于类型的一对id:slots。
\end{itemize}

这取决于如何使用和内部分布资源组的测试实现。

编写和运行测试显然是提高代码质量的主要方式。然而,另一个有趣的指标通常是测试实际覆盖了多少代码。调查和报告代码覆盖率可以提供有趣的提示,这不仅是关于软件测试的范围,还关乎于差异。


















